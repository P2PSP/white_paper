\label{sec:TTS}

\begin{figure}
  \fig{600}{6cm}{icecast-P2PSP} \caption{A possible data-flow in an
  hybrid Icecast/P2PSP network. $S$ represents a splitter and $P_i$ a
  peer.\label{fig:icecast-P2PSP}}
\end{figure}

% On channels ...
P2PSP supposes that there is a collection of channels that are
broadcasted in parallel.\footnote{For example, channels in P2PSP is
  similar to channels in DVB (Digital Video Broadcasting): streams of
  media that are transmitted in parallel with the rest channels.} The
channels are available at one or more\footnote{Most client/server
  streaming services provide the possibility of building a CDN
  (Content Delivery Network), a tree of servers (sources) sending the
  same contents.} streaming servers, and each channel has a different
URL (Universal Resource Locator), usually expressed as a HTTP address.

\begin{figure*}
  \fig{900}{9cm}{TTS} \caption{Procedures to determine a suitable team
    given a channel.\label{fig:TTS}}
\end{figure*}

% The channels tracker
When a user runs a P2PSP's media player, it connects to a channels
tracker (see Fig.~\ref{fig:TTS}). The tracker returns the list of
channels currently broadcasted, and the user must chose one of
them. This information is sent back to the channels tracker, which
responds with the end-point of a teams tracker. There is exactly one
teams tracker per channel.

% The teams tracker
Then, the player connects to the teams tracker, that should returns a
list end-point of splitter, with space in their teams. If none of the
splitters that are broadcasting the channel has room for the incoming
peer, the teams tracker will create a new empty team\footnote{An empty
  team is built with a splitter and at least one monitor peer.}
connected to a source\footnote{Notice that we are supposing that at
  least a source can handle a new listener. If this is not true, the
  splitter will not be able to connect to a source, it will inform to
  the teams tracker, which will inform to the player, which should
  inform to the user with the warning ``The maximum number of
  concurrent users has been reached for this channel. Please, wait or
  select a different channel''.} of the channel, and will return the
splitter's end-point.

% Searching the closest team 
With this information (a list of splitters with room in their teams),
the player connects with all the splitters in parallel and the fastest
connection determines the final selected splitter. The rest of
connections are closed. This procedure should select the ``closest''
splitter to the player in terms of network latency.

% Spawning the peer
Finally, the player spawns a peer and the team-attaching process, as
described in Sec.~\ref{sec:DBS}, starts.

% As can be seen in Fig.~\ref{fig:TTS}, to find a team given a
% channel, several entities must run different tasks. The player,
% controlled by the user, must obtain a splitter's end-point. In the
% channels tracker, a task listens to the players and serves, first
% the list of current channels and second, a teams tracker dedicated
% to such channel. Two tasks in the teams tracker, one for serving the
% splitters and another for keeping updated the list of splitters with
% room in its teams, must be run.
