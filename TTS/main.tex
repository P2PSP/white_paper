\label{sec:TTS}
\begin{figure*}
  \fig{800}{8cm}{TTS} \caption{Entities and procedures involved in the
    process of determing a splitter (i.e. a team with room for an
    incoming peer) for a given channel.\label{fig:TTS}}
\end{figure*}

P2PSP supposes that there is a collection of channels that are
broadcasted in parallel.\footnote{The concept of channels in P2PSP is
  similar to DVB (Digital Video Broadcasting): streams of media that
  are transmitted in parallel with the rest channels.} The channels
are generated by one or more\footnote{Most client/server streaming
  services provide the possibility of building a CDN (Content Delivery
  Network), a tree of servers (sources) sending the same contents.}
streaming servers, and each channel has a different URL (Universal
Resource Locator), usually expressed as a HTTP address.

When a user runs a P2PSP's media player, it connects to a channels
tracker (see Fig.~\ref{fig:TTS}). The tracker returns the list of
channels currently broadcasted, and the user must chose one of
them. This information is sent back to the channels tracker, which
responds with the end-point of a teams tracker. Then, the player
connects to the teams tracker, that returns the end-point of a
splitter, with space in its team. If none of the splitters that are
broadcasting the channel has room for the incoming peer, the teams
tracker will create a new empty team connected to a
source\footnote{Notice that we are supposing that the source can
  handle a new listener. If this is not true, the splitter will not be
  able to connect to a source, it will inform to the teams tracker,
  which will inform to the player, which should inform to the user
  with a ``The maximum number of users for this channel has been
  reached'' message.} of the channel, and will return the splitter's
end-point.

With this information (where a splitter should be listening to the
peers of its team), the player spawns a peer and the attaching process
starts. As can be seen in Fig.~\ref{fig:TTS}, to find a team given a
channel, several entities must run different tasks. The player,
controlled by the user, must obtain a splitter's end-point. In the
channels tracker, a task listens to the players and serves, first the
list of current channels and second, a teams tracker dedicated to such
channel. Two tasks in the teams tracker, one for serving the splitters
and another for keeping updated the list of splitters with room in its
teams, must be run.
