% Emacs, this is -*-latex-*-

\label{sec:IMS}

\begin{note}
  Finished and implemented.
\end{note}

IMS modifies the functionality of DBS.

IPM is available by default in LANs (Local Are Network)s and VLANs
(Virtual LANs)~\cite{shabtay2011ip}, but not in the
Internet~\cite{Comer1}. IMS runs on the top of DBS and provides
efficient native IPM, where available.

All peers in the same LAN or VLAN have the same network address. When
a joining peer $P_i$ receives the list of peers from its splitter,
first checks if there are neighbors in the same subnet. For all those
peers, $P_i$ uses the IP address $\mathtt{224.0.0.1}$ (all systems on
this subnet), (default) port $\mathtt{1234}$, to multicast (only) the
chunks received from the splitter. Therefore, all peers in the same
local network communicate using this multicast group address and
port. The rest of external peers will be referenced using their public
end-points.

\begin{comment}
\subsubsection{Listening IPM}
\label{imp:listening}
By default, all peers listen to $\mathtt{224.0.0.1}$ (All Systems on
this Subnet), port $\mathtt{1234}.

\subsubsection{Joining the team}
\label{ims:joining}
Two peers in the same LAN share the network address. When an incomming
peer ${\cal P}_i$ receives the list of peers, checks if there are
peers in the same LAN. For those peers, ${\cal P}_i$ use the end-point
$\mathtt{224.0.0.1:1234}$.

In IMS, peers
star joining the team as in DBS. However, when a ${\cal P}_i$ receives
a $[\mathtt{hello}~{\cal P}_i]$ from an incomming ${\cal P}_N$, ${\cal
  P}_i$ checks if the source IP address corresponds to the same
LAN. If this is true, ${\cal P}_i$ replies with a
$[\mathtt{hello}~{\cal P}_x]$, where ${\cal P}_x$ is the end-point of
a multicast group selected by ${\cal P}_i$ (if this is not the first
time ${\cal P}_i$ receives a hello from the same LAN, the previous
${\cal P}_x$ is selected). Then ${\cal P}_N$ sends a
$[\mathtt{neighborhood\_request}~{\cal P}_x]$ to ${\cal P}_x$ that, if
received by ${\cal P}_i$, replies with a
$[\mathtt{neighborhood\_accept}~{\cal P}_x]$ to ${\cal P}_x$. Notice
that ${\cal N}({\cal P}_i)$ remain constant (except in the first IMS
iteration where ${\cal P}_x$ is added) and that ${\cal N}({\cal
  P}_N)\cup={\cal P}_x$. None message is sent from 

\subsubsection{Leaving the team}
\label{ims:leaving}

A incomming peer ${\cal P}_N$ 

% What is P2PSP and what makes it special
In this context, we propose P2PSP (Peer-to-Peer Straightfoward
Protocol), an application-layer multicast solution based on the P2P
communication paradigm. Like IP multicasting, in P2PSP it holds that:
(1) there is only a source of data (the so called \emph{splitter}) and
a dynamic group of destinations (the \emph{peers}), (2) only one copy
of the data is sent by the splitter, (3) a ``best-effort''
transmission service is provided, and (4) the nodes (splitter and
peers) do not have knowledge about the transmitted content. Unlike IP
multicasting, in P2PSP: (1) the splitter is the only source, (2) the
nodes know each other, and (3) the network infrastructure only needs
to provide a unicast transmission service (multicasting is achieved by
the nodes at the application-layer level building a overlay
network). However, notice that as a consecuence of that the
communication model used in both systems is virtually the same, P2PSP
can use IP multicasting when this facility is available. As it will
be shown in the evaluation of P2PSP, both protocols (P2PSP and IP
multicast) have a similar performance in terms of bandwidth and
latency.

IMS defines the behaviour of those nodes that can communicate between
them using IP multicast.

\subsubsection{Using IP multicast}
By definition, peers that share the same network address (for example,
peer that are behind the same NAT device) can use IP multicast to
communicate. One of the peers of the multicast group (the oldest one)
works as a bridge between the ``unicast'' and the ``multicast''
subteams. When this peer receives a [\textsf{hello}] from a new peer
that is in the same subnetwork, it selects a multicast channel (IP
address and port) and sends this information to the incomming
peer. Only the bridge peer (the first peer of the ``multicast
subteam'' in joining) is taken into account for the rest of the team
and the splitter. Only the bridge receives chunks from the splitter
and its ``unicast'' neighbors, and this peer is in charge of
retransmitting them to the multicast group.

When leaving, ``multicast'' peers only need to send a
[\textsf{goodbye}] to the splitter and the bridge, which is the only
peer in their lists. The bridge store in its list of peers the
multicast channel, but the \textsf{chunk\_debt} counter is never
incremented.

\subsubsection{Chunk buffering}
Each peer stores the received chunks in a buffer before play them.
The buffer size $B$ in chunks is equal to the expectable (physical)
network's jitter. If $V$ is the average bit-rate of the media
(measured in bits/second), the buffering time is
\begin{equation}
  T = C/V*B.
  \label{eq:buffering_time}
\end{equation}

\end{comment}
