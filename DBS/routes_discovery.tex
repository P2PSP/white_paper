% Emacs, this is -*-latex-*-

% Routes Discovery and Topology Optimization

\label{sec:routes_discovery}

Chunks can be lost under bandwidth and buffering time constraints. A
chunk is lost when it is time to send it to the player, i.e. when it
is pointed by $p_p$, and the chunk has not been received.  When a peer
realizes that a chunk pointed by $p_p$ has been lost, nothing can be
done to recover it. In order to avoid this, peers pre-fetch
``potentially lost'' chunks at the buffer position $p_p+p_h$, where
$p_h\geq 0$ is the pre-feching horizon. Setting $p_h=0$, the
pre-fetching is disabled and only those chunks that really are lost
will be requested.  On the contrary, the higher the $p_h$, the more
aggressive the pre-fetching is.

% \leorem{Se podría ir eliminando a aquellos que no han respondido o a
% los que se les ha mandado prunning de una posible lista de donde
% coger el peer aleatorio.} -> Supongo, pero ni idea de cómo rendiría
% esto. Habría que implementarlo y experimentar.

For each (potentially) lost chunk with number
$\text{lost\_chunk\_number}$, peers send a
$[\mathtt{request}~\text{lost\_chunk\_number}]$ message to a random
peer of the team. It may happens the requested peer has not the
chunk. In such case, the chunk will not be recovered. An alternative
to the random selection is shown in Sect. \ref{sec:FCS}.

When a peer $P_i$ receives a request message from $P_j$, $P_i$ adds
$P_j$ to $\mathtt{forward}[P_k]$, where $P_k$ is the origin peer of
the chunk stored in the position
$(\text{lost\_chunk\_number}~\mathit{mod}~2B)$ of $P_i$'s buffer in
case this chunks has been received (otherwise, the request is
ignored). \leo{This produces peer $P_i$ continues forwarding $P_x$'s
  chunks to $P_j$ in next rounds}.

\leo{Due to the random selection of the destination of} the request messages, redundant routes can be created. Therefore, some chunks could be received more than once. To avoid future chunk duplicates, a peer send a pruning message to a peer \leo{that sends a duplicated chunk}. The receiver of the pruning message counts the number of times that a origin peer has
been \leo{asked to be} pruned. When this counter is higher than a threshold $T$ (the
maximum number of generated duplicates) the corresponding entry in the $\text{forward}[]$ table is deleted.
\leorem{No bastaría con quitar el par que envía el prunning del forward para ese origen?}{\color{red} No te comprendo bien. Eso es justamente lo que hacemos.}

Now, we can define more accurately the \gls{neighborhood-degree} (see
Sec.~\ref{sec:chunk_flooding}) as the number of different destination
peers for each possible origin that a peer forwards. For example, if a
peer $P_i$ forwards chunks from the origin $P_i$ to 10 neighbors, the
neighborhood degree of $P_i$ for the origin $P_i$ is 10, and if the
peer $P_i$ also forwards chunks from an origin $P_j$ to 5 neighbors,
the neighborhood degree of $P_i$ for the origin $P_j$ is 5.

Considering the rules described before, the neighborhood degrees of
peers can decrease or increase to minimize the number of (potentially) lost chunks. An increment in the degree for the origin of a requested
chunk with $\text{lost\_chunk\_number}$ in $P_i$ is produced when $P_i$
receives a $[\mathtt{request}~\text{lost\_chunk\_number}]$ from a peer
that is not a neighbor, yet. \leorem{Podría ser un vecino (conectado) al que no le lleguen los chunks del origen. No entiendo porqué no tiene que ser un vecino.}{\color{red} Porque si fuera ya un vecino para ese origen, el grado de vecidad no aumentaría} On the contrary, a decrement in the
degree for the origin of a pruned chunk with 
$\text{duplicate\_chunk\_index}$ in $P_i$ is produced when $P_i$
receives a $[\mathtt{prune}~\text{duplicate\_chunk\_index}]$ from a
neighbor peer, for that origin. \leo{The origin could also be $P_i$ itself in case a neighbor of $P_i$ forwards a requested chunk with origin $P_i$ before the chunk of $P_i$ reach the destination peer}. {\color{red} Esta \'ultima frase es m\'ia? (no la entiendo)}

The continued reception of pruning and requesting messages at peer $P_i$ produces the jagged array
$\mathtt{forward}[]$ gets shorter (smaller \gls{neighborhood-degree}s) and larger, respectively.
