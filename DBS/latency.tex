% Emacs, this is -*-latex-*-

% Latency of the streaming system

\label{sec:latency}

%\begin{figure}
%  \centering
%  \vbox{\myfig{graphics/ring}{6cm}{400}}
%  \caption{A ring-shaped overlay.}
%  \label{fig:ring}
%\end{figure}

The start-up time $t_l$ is the latency that the end-user experiments
during the redering of the media. $t_l$, defined as
\begin{equation}
  \label{eq:t_l}
  t_l = t_p + t_b,
\end{equation}
is the sum of the physical
delay $t_p$ and the buffering time $t_b$, where
\begin{equation}
  \label{eq:t_b}
  t_b = Bt_c,
\end{equation}
is the time that has elapsed since the peer receives two chunks
distanced $B$ positions in the buffer, being $B$ the buffer size (in
chunks), and $t_c$ the chunk time (considering $t_c$ a
constant\footnote{If $R$ were time-variying, $t_b$ would be the sum of
  the $t_c$ for each received chunk.}). In other words, the buffering
time is the delay in the rendering of the media required to expect a
seamless playing. Notice that Eq.~\ref{eq:t_b} does not assume that
the chunks fill the buffer in order during the buffering time.

To minimize $t_l$, $t_p$ and $t_b$ should be as small as possible. How
to minimize $t_p$ is out of the scope of this discussion. However, to
minimize $t_b$, we can reduce $t_c$ and/or $B$, but these actions can
carry negative consequences. On the one hand, $t_c$ should be large
enough to reduce the overhead of the headers used by the transport
protocol. On the other hand, if $B$ is too small (for instance $B<N$),
the peer will not have enough space to buffer all the chunks of a
round, causing that incoming chunks to overwrite others before they
can be played due to a low probability of receiving chunks in
order. This problem can also happen even if $N\leq B<2N$, because the
protocol jitter, $\Delta t_b$, that a chunk can experiment is the sum
of the jitter produced by the splitter for this peer, that can as high
as $Nt_c$, and the jitter produced by the flooding of the chunk to the
rest of the team, that also can be has high as $Nt_c$. Notice that
this jitter is the same for two extreme topologies of the overlay: (1)
a full-connected mesh (Fig.~\ref{fig:full}), or (2) a ring
(Fig.~\ref{fig:ring}), when the chunks travel only in one
direction\footnote{Notice that if chunks can travel in both
  directions, we whould have a tree.}. Therefore, in the case of a full-connected mesh, peers have to use a
\begin{equation}
  \label{eq:minimum_B}
  B\ge 2N^*
\end{equation}
in order to avoid the loss of chunks due to high $\Delta t_p$. We will
refeer to $B=2N^*$ as the critical buffer size.

% As a rule of thumb, the larger the buffer size the lower the
% probability of losing chunks, as a consequence of a high $\Delta
% t_p$ (physical jitter).

Lets analyze the buffering time $t_b$ in two interesting cases. Lets
suppose that the number of the first received chunk is $x_1$ and that
the rest of chunks of the buffer of size $B$ are received\footnote{All
  the chunks received with an number smaller than $x_1$ will be
  discarded. Notice also that chunks can arrive in any order.}, being
the chunk $x_{1+B}$ the last one (the best scenario). In this case,
the $t_b$ can still be estimated by Eq.~\ref{eq:t_b}. Lets imagine now
that after receiving $x_1$ the chunk $x_{1+B}$ is received (chunks
$x_2, \cdots x_{1+B-1}$ have been lost or delayed too much), and the
playing starts (probably, with severe artifacts). In this case, $t_b$
also corresponds to Eq.~\ref{eq:t_b}, because the chunk $x_{1+B}$ was
generated $B$ chunk times after $x_1$, and we cannot wait more during
the buffering time if we want to dimension $t_b$. After considering
these two extreme situations, we can deduce that the start-up time
$t_s$ does not depend on the chunk loss ratio during the
buffering-time (for loss ratios smaller than one), but only on $B$,
$t_c$ and $t_p$.

Finally, notice that the time that passes from an incoming peer
request to join a team to a splitter until it accepts the peer, and
the buffering time that the player can require, have not been
considered in this discussion. However, both times should be, in
general, significantly lower than $t_b$, that in the case of P2PSP
corresponds to the $t_s$ that end-users experiment.

