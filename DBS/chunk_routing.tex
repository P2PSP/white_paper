\label{sec:chunk_routing}
It can happen\footnote{Especially if $P_k$ has just joined to the
  team.} that the received chunk by peer a $P_k$ from a neighbor peer
$P_i$ is a duplicate chunk. In this case, $P_k$ sends to $P_i$ a
$[\mathtt{prune} x]$ message, where $x$ is number of the duplicate
chunk. Thus, only the first neighbor $P_j$ to send to $P_k$ a chunk
$x$ ``originated'' for example at a peer $P_l$, will do that in the
future (remember that the origin peers are stored in the chunks
messages), at least that $P_k$ revokes this routing information by
sending a $[\mathtt{request} x]$ to one or more (possibly the rest of)
peers of $T^k$. Notice that, this process will be succesful only if
$P_i$ have received the chunk $x$. If this has not happen, this will
not have effect. However, notice that in this case this procedure will
be repeated with the reception of the following chunk originated at
$P_l$, and that $P_i$ should use the same procedure if chunk $x$ is
missing.

It can also happen that some chunks are not received on time (remember
that chunks are transmitted using UDP which is unreliable). A chunk is
considered as lost by peer $P_k$ when it is time to send it to the
player and the chunk has not been received. In this moment, as in the
previous situation, $P_k$ sends a $[\mathtt{request} x]$ to one or
more (possibly the rest of) peers of $T^k$, where $P_m$ is, where
$x=y+\Delta$, being $y<x$ the number of closest received chunk to $x$,
and $\Delta$ is the distance in chunks between $x$ and $y$.
