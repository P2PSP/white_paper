% Emacs, this is -*-latex-*-

% Team Definition

\label{sec:team_def}

For the splitter, a team is a set of one or more peers (referenced by
their end-points) that share the same stream. By definition, in a team
of size one (the corresponding splitter is considered out of the team
that it feeds), the only peer is known as a \gls{monitor} peer, and in
a team with more than one peer, at least one of them must be a
monitor. Monitors are \leochange{instantiated}{deployed} by the overlay administrator to
check different aspects of the broadcasting, such as, the quality of
the rendered video at the peers or the end-user latency.

The number of peers (normal peers and monitors) in a team has a
maximum $N^*$ (see Tab.~\ref{tab:DBS_nomenclature}). This parameter
has an impact on the latency of the protocol (see
Sec.~\ref{sec:buffering_chunks}), and usually is defined by the
administrator of the overlay.

\begin{table}[hbt]
  \centering
  \begin{tabular}{rl}
    Parameter & Meaning \\
    \hline
    $N^*$ :  & Maximum number of peers in a team. \\
    $C$ :    & Chunk size. \\
    $B$ :    & Buffer size, in chunks, in the peers. \\
    $B'$ :   & Length of the list of the last $B'$ peers served by the splitter. \\ 
    $D$ :    & Diameter of the flooding tree. \\
    $L^*$ :  & Maximum allowed number of lost chunks. \\
    $M$ :    & Number of monitors. \\
    $R$ :    & Average bit-rate of the media. \\
    Variable & \\
    \hline
    $N$ :    & Number of peers in the team. \\
    $t_c$ :  & Chunk time. \\
    $t_r$ :  & Round time. \\
    $t_b$ :  & Buffering time. \\
    $t_p$ :  & Physical network latency. \\
    $t_s$ :  & Start-up time.
  \end{tabular}
  \caption{Nomenclature used in DBS.} %
  \label{tab:DBS_nomenclature}
\end{table}

