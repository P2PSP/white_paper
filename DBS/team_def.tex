% Emacs, this is -*-latex-*-

% Team Definition

\label{sec:team_def}

A team is a set of one or more peers (referenced by their end-points)
that share the same stream. By definition, in a team at least one peer
is a \gls{monitor} (peer). Monitors are deployed by the overlay
administrator to check different aspects of the broadcasting, such as,
the expeted quality of the rendered video at the peers or the
estimated end-user latency.

The number of peers (normal peers and monitors) in a team has a
maximum $N^*$ (see Tab.~\ref{tab:DBS_nomenclature}). This parameter
has an significative impact on the maximum latency of the protocol (see
Sec.~\ref{sec:buffering_chunks}), and is usually defined by the
administrator of the overlay.

\begin{table}[t]
  \centering
  \begin{tabular}{rl}
    Parameter & Meaning \\
    \hline
    $N^*$ :  & Maximum number of peers in a team \\
    $C$ :    & Chunk size \\
    $B$ :    & Buffer size, in chunks, in the peers \\
    $B'$ :   & Length of the list of the last $B'$ peers served by the splitter \\ 
    $D$ :    & Diameter of the flooding tree \\
    $L^*$ :  & Maximum allowed number of lost chunks \\
    $M$ :    & Number of monitors \\
    $R$ :    & Average bit-rate of the media \\
    $H$ :    & Pre-fetching horizon \\
    Variable & \\
    \hline
    $N$ :    & Number of peers in the team \\
    $t_c$ :  & Chunk time \\
    $t_r$ :  & Round time \\
    $t_b$ :  & Buffering time \\
    $\Delta t_b$ : Protocol (buffering) jitter \\ 
    $t_p$ :  & Physical (network) latency \\
    $\Delta t_p$ : Physical jitter \\
    $t_s$ :  & Start-up time \\
    $i_p$ :  & The played chunk index 
  \end{tabular}
  \caption{Nomenclature used in DBS.} %
  \label{tab:DBS_nomenclature}
\end{table}

