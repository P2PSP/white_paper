%%% Local Variables:
%%% mode: latex
%%% TeX-master: "<none>"
%%% End:

\label{sec:free_riding_control}

Monitor peers (which are trusted) complain to their splitter with a
$[\mathtt{lost}~\text{lost\_chunk\_index}]$ for each lost chunk. The
splitter only considers these type of messages if they come from a
monitor.

If a peer $P_o$ accumulates more than $L^*$ (see
Tab.~\ref{tab:DBS_nomenclature}) losts in $R$ rounds, $P_o$ is removed
from the splitter's list of peers. All peers do the same, but in this
case they also remove the selfish neighbors from the $\text{forward}$
and $\text{pending}$ tables.

%Notice that $L$ will
%tend to be proportional to the number $M$ of monitors, especially if
%those cases where $P_o$ is a gone peer that was unable to transmit the
%$[\mathtt{goodbye}]$ messages.

\begin{notex}
This last functionality has not been implemented, at least, as it has
been explained here. The forget() thread is controlled by a timer, not
by a counter of rounds.
\end{notex}

%Peers also control that at least one chunk is received from a neighbor
%in each round.\footnote{Peers recognize that a new round has started
%  when a new chunk is received from the splitter.} If happens that a
%peer $P_x$ does not receives a chunk from peer $P_y$ between $D^*$
%consecutive rounds, $P_x$ removes $P_y$ of its forwaring table.
