% Emacs, this is -*-latex-*-

% Free-riding Control at the Splitter

\label{sec:free_riding_control}

The splitter remembers which chunk, of a list of the last $B'$
transmitted chunks, was sent to each peer of the team. Notice that, in
order to remember the chunk that was sent to each peer in each round,
it must be hold that $B'\ge N$. \note{See
  \href{https://github.com/P2PSP/simulator/blob/f0c73be1817e7d3b816cc61cd2c8e59b17f9a0e6/src/core/splitter_dbs.py\#L296}{$\text{destination\_of\_chunk}[]$
    in \texttt{splitter\_dbs.py}}.}

Monitor peers (which are trusted peers) complain to their splitter
with a $[\mathtt{lost}~\text{lost\_chunk\_number}]$ for each lost
chunk. The splitter only considers these type of messages if they come
from a monitor.

%Notice that $L$ will
%tend to be proportional to the number $M$ of monitors, especially if
%those cases where $P_o$ is a gone peer that was unable to transmit the
%$[\mathtt{goodbye}]$ messages.

\begin{notex}
This last functionality has not been implemented, at least, as it has
been explained here. The forget() thread is controlled by a timer, not
by a counter of rounds.
\end{notex}

%Peers also control that at least one chunk is received from a neighbor
%in each round.\footnote{Peers recognize that a new round has started
%  when a new chunk is received from the splitter.} If happens that a
%peer $P_x$ does not receives a chunk from peer $P_y$ between $D^*$
%consecutive rounds, $P_x$ removes $P_y$ of its forwaring table.
