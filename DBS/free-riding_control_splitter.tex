% Emacs, this is -*-latex-*-

% Free-riding Control at the Splitter

\label{sec:free_riding_control}

Monitor peers (which are trusted peers) complain to their splitter
with a $[\mathtt{lost}~\text{lost\_chunk\_number}]$ for each lost
chunk that they detect. The splitter only considers these type of
messages if they come from a monitor. FCS (see Sec. \ref{sec:FCS}) is
used to control free-riders at peers.

The splitter remembers which chunk, of a list of the last $B'>N^*$
transmitted chunks, was sent to each origin peer of the team. The
splitter also counts of lost chunks for each peer of the team. The
corresponding counter is incremented for each lost chunk reported by a
monitor and, if the counter of a peer goes over a threshold $L^*$, the
corresponding peer is removed from the list of peers, and the rest of
counters are reset. Notice that the smaller the $L^*$, the higher the
belligerence with the selfish peers. The selection of $L^*$ should be
also proportional to the number of monitor peers $M$.
