In each team there are a set of $M^j\leq N$ \emph{monitors} ${\cal
M}^j_i\in {\cal T}^j$ (trusted peers whose behavior is predictable and
that are only known by the splitter), which complain to their splitter
${\cal S}^j$ when a chunk is lost. For every sent chunk $x$ to ${\cal
P}^j_k$, ${\cal S}^j$ performs $D[x]={\cal P}^j_k$ to associate every
sent chunk with its destination (origin) peer ${\cal P}^j_k$. If
happens that $L$ $[\mathtt{LC}~x]$ ($\mathtt{L}$ost $\mathtt{C}$hunk)
messages have beeen received from any of the monitors (the splitter
only considers these kind of messages if they come from a monitor) in
a number of rounds (see Fig.~\ref{fig:free_riding_control_splitter}
for the exact computation of the number of lost chunks per peer per
round), then ${\cal P}^j_k$ is considered as a gone peer and it is
removed from the list of peers of the splitter.

\note{This last functionality has not been implemented, at least, as it has been explained here. The \texttt{forget} thread is controlled by a timer, not by a counter of rounds.}
