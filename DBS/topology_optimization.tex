When a peer $P_i$ is joining the team, it sends a $[\mathtt{hello}]$
to each other peer of the team, which will add $P_i$ to their
forwarding and pending tables. So, in absence of communication
constraints, the team will be organized as a full-connected overlay.

Peers forward chunks to their neighbors in the order in which the
entries in $\mathtt{pending}[]$ table is accessed, that depends on the
$\mathtt{supportivity}[]$ table. In each round, peers serve first to
the supportive neighbors, fact that will increase their supportivity
in the supportive neighbors, and viceversa. Unsupportive neighbors are
deleted from $\mathtt{forward}[]$ and $\mathtt{pending}[]$, decreasing
the neighborhood degree.

The neighborhood degree can also grow. Chunks are lost under bandwidth
and buffering time constraints. When a chunk is lost, the peer
requests to receive the rest of chunks from the corresponding origin
peer to a random peer of the known team. If duplicates are generated,
a prune message will remove the slower route from that origin peer,
generating that the most reliable (and possiblely faster) route (that
loss less chunks) to endure. When this happens, the requesting peer
will be added to the tables $\mathtt{forward}[]$ and
$\mathtt{pending}[]$ of the requested peer, increasing its
neighborhood degree.

