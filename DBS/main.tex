%%% Local Variables:
%%% mode: latex
%%% TeX-master: "<none>"
%%% End:

\label{sec:DBS}

\begin{table}
  \begin{tabular}{rl}
    Parameter & Meaning \\
    \hline
    $B$    & Buffer size in chunks \\
    $C$    & Chunk size \\
    %$D^*$  & Maximum chunk debt between peers \\
    $L^*$  & Maximum allowed number of lost chunks \\
    $M$    & Number of monitors \\
    $N^*$  & Maximum number of peers in a team \\
    $R$    & Number of rounds to compute $L^*$.
  \end{tabular}
  \caption{Nomenclature used in DBS.\label{tab:DBS_nomenclature}}
\end{table}

DBS provides ALM~\cite{banerjee2002scalable} of a media stream in
unicast environments~\cite{comer2003computer}. The media is sent by a
streaming \emph{server}, and received by a \emph{splitter} \note{(see
  Sec.~\ref{sec:LBS})}. The splitter divides the stream into a
sequence of \emph{chunks} of data, and relay them to its \emph{team}
using a round-robing schema, where it can be up to $N^*$ \emph{peers}
(see Tab.~\ref{tab:DBS_nomenclature}). Each peer gathers the chunks
from the splitter and the rest of peers of the team, and sends them to
at least one \emph{player}\footnote{Peer should run even if no
  player(s) are connected to it.}.

\begin{comment}
In single layered streams\footnote{Each layer of a
  scalable stream is received by a different peer attached to the same
  player capable or render scalable media.}, each peer is spawned by a
player (normal users should not run peers directly).
%This system has been described in Fig.~\ref{fig:DBS_system}.
\end{comment}

\begin{comment}
/* quitar: We define the set of teams as
$\{T\}$,
%=\{T^1,\cdots,T^{|T|}\}$,
and enumerate the peers in the team $T$ as $T=\{P_1,\cdots,P_{|T|}\}$. */
\end{comment}
