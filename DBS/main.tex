% Emacs, this is -*-latex-*-

% Data Broadcasting Set of rules (main)

\label{sec:DBS}

\begin{table}
  \centering
  \begin{tabular}{rl}
    Parameter & Meaning \\
    \hline
    $N^*$  & Maximum number of peers in a team \\
    $C$    & Chunk size \\
    $B$    & Buffer size in chunks in the peers \\
    $B'$   & Size of the set of last $B'$ peers served by the splitter \\ 
    %$D^*$  & Maximum chunk debt between peers \\
    $L^*$  & Maximum allowed number of lost chunks \\
    $M$    & Number of monitors \\
    $R$    & Number of rounds to compute $L^*$. \\
    Variable & \\
    \hline
    $N$    & Number of peers in the team \\
    $t_C$  & Chunk time \\
    $t_R$  & Round time \\
    $t_B$  & Buffering time \\
    $L$    & Physical network latency \\
    $T$    & Latency experimented by the end-user
  \end{tabular}
  \caption{Nomenclature used in DBS.} %
  \label{tab:DBS_nomenclature}
\end{table}

DBS provides ALM~\cite{banerjee2002scalable} of a media stream in
unicast environments~\cite{comer2003computer}. First, the media is
sent by a streaming \emph{server}, and received by a \emph{splitter}
\note{(see Sec.~\ref{sec:LBS})}. Then, the splitter divides the stream
into a sequence of chunks of data, and relay them to its \emph{team}
using a round-robing schema. A \emph{team} is composed by peers and
each peer gathers the chunks from the splitter and the rest of peers
of the team, and sends them to at least one
\emph{player}\footnote{Peer should run even if no player(s) are
  connected to it.}.

\begin{comment}
In single layered streams\footnote{Each layer of a
  scalable stream is received by a different peer attached to the same
  player capable or render scalable media.}, each peer is spawned by a
player (normal users should not run peers directly).
%This system has been described in Fig.~\ref{fig:DBS_system}.
\end{comment}

\begin{comment}
/* quitar: We define the set of teams as
$\{T\}$,
%=\{T^1,\cdots,T^{|T|}\}$,
and enumerate the peers in the team $T$ as $T=\{P_1,\cdots,P_{|T|}\}$. */
\end{comment}
