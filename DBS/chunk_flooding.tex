%
% Chunk flooding
%

When a peer ${\cal P}^j_k$ receives a chunk from ${\cal P}^j_i$,
${\cal P}^j_k$ floods the chunk to ${\cal T}^j \setminus {\cal
P}^j_i$, using again a round-robin schema (see
Fig.~\ref{fig:chunk_generation_and_flooding}). Besides, if it is a
duplicate chunk, ${\cal P}^j_k$ sends to ${\cal P}^j_i$ a
$[\mathtt{NRFCF}~{\cal P}^j_l]$ ($\mathtt{N}$ot $\mathtt{R}$elay
$\mathtt{F}$uture $\mathtt{C}$hunks $\mathtt{F}$rom) message, where
${\cal P}^j_l$ is the origin peer of the duplicate chunk. Thus, only
the first ``neighbor'' ${\cal P}^j_i$ to send to ${\cal P}^j_k$ a
chunk ``originated'' at ${\cal P}^j_l$ will do that in the future, at
least that ${\cal P}_k$ revokes this routing information by sending
$[\mathtt{RFCF}~{\cal P}^j_l]$ ($\mathtt{R}$elay $\mathtt{F}$uture
$\mathtt{C}$hunks $\mathtt{F}$rom) to one or more (possibly the rest
of) peers of ${\cal T}^j_k$.

%When peers receive chunks from their splitter, they must flood them to
%their neighbors until the chunks are broadcasted to the whole team
%(Fig.~\ref{fig:chunk_generation_and_flooding}). Lets suppose that
%${\cal P}_k$ receives a chunk. In the case the sender is its splitter,
%${\cal P}_k$ floods the chunk to $N({\cal P}_k)$. However, if the
%sender is a peer ${\cal P}_m\in N({\cal P}_k)$, ${\cal P}_k$ adds
%${\cal P}_m$ to $N({\cal P}_k)$ if ${\cal P}_m$ is a new neighbor, and
%forwards the chunk to the rest of its neighborhood ${\cal P}_n\in
%N({\cal P}_k)\setminus{\cal P}_m$ if ${\cal P}_k$ is in the shortest
%between ${\cal P}_n$ and the origin peer ${\cal P}_i$ of the relayed
%chunk. This will be true if ${\cal P}_k$ is the gateway of ${\cal
%  P}_n$ to go from ${\cal P}_n$ to ${\cal P}_i$. Therefore, a flooding
%with prunning based on shortest path routing is used.
