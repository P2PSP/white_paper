% Emacs, this is -*-latex-*-

% Buffering chunks

\label{sec:buffering_chunks}

In order to hide the jitter generated by the physical network and the
protocol itself, peers need to store the received chunks in a buffer
during a period of time, before sending them to a player. A chunk with
number $x$ is inserted in the position $(x~\mathit{mod}~2B)$ of the
buffer, where $B$ is the maximum number of chunks that the buffer
stores. In a peer's life, $B$ is a constant specified by the user,
but it is not compulsory that all peers of a team use the same buffer
size.

The buffer is implemented as a circular queue of $2B$ chunks, what is
filled with up to $B$ chunks during the \gls{buffering-time} $t_b$,
which is the main part of the \gls{start-up-time} experienced by
users. Chunks with a higher number (newer chunks) are inserted near of
(depending on the order in which the chunks arrive to the peer) the
head of the buffer. The (received) chunks pointed by the tail of the
buffer $p_p$ (the playing pointer) are sent to the player. This action
is carried out each time a new chunk is received\footnote{DBS does not
  know anything about the content and therefore, about the timing of
  the chunks.}. During the playing process, empty cells in the buffer
(caused by the chunks that have not been received on time) are skipped.

% Hablar de la relación entre B y el tamaño del team. Tal vez, cuando
% se presente la expresión de la latencia en función del grado de
% conectividad. En el caso extremo en que todos los peers se
% conectaran con todos, B >= N^*, el número máximo de peer en el team.
