\begin{table}
  \begin{tabular}{rl}
    Parameter & Meaning \\
    \hline
    $G$       & Maximum number of teams in a channel \\
    $N$       & Maximum number of peers in a team \\
    $C$       & Maximum size of a chunk of media \\
    $B$       & buffer size in chunks \\
    ~\\
    Entity  & Role \\
    \hline
    $O$     & the media source \\
    $\{T\}$ & all the teams related to the same media channel \\
    $R$     & teams tracker (relates channels with teams)\\
    $\{S\}$ & the set of splitters broadcasting the same channel \\
    $T$     & one of the $|T|$ teams in $\{T\}$ \\
    $S$     & the splitter of $T$ \\
    $P_i$   & $i$-th peer of $T$ \\
    $T_i$   & the neighbors of $P_i$ \\
    ~\\
    Variable            & Meaning \\
    \hline
    $1\leq |\{T\}|\leq G$ & current number of teams broadcasting the same channel \\
    %$1\leq |S|\leq G$   & current number of splitters related with $T$ \\
    $1\leq |T|\leq N$     & current number of peers in team $T$ \\
    $1\leq |T_i|\leq N$   & current number of neighbor peers in $T_i$ \\
  \end{tabular}
  \caption{Nomenclature used in this
    documentation.\label{tab:nomenclature}}
\end{table}

The nomenclature used for describing P2PSP is shown in
Table~\ref{tab:nomenclature}. In this documentation, only one media
channel has been considered, therefore, if more channels are needed,
all these objects should be instantiated for each different channel
(for example, there should be a different tracker for each
channel). In P2PSP, a channel is the equivalent to a same concept, for
example, in the DVB (Digital Video Broadcasting) system.
