% Emacs, this is -*-latex-*-

\label{sec:ACS}

\begin{note}
  Unfinished.
\end{note}

ACS modifies the functionality of DBS.

Basically, ACS relaxes the peer's uploading requirements imposed by
DBS. ACS is based on the idea of using the information that the
splitter knows about the number of chunks that each peer has lost (see
Sec~\ref{sec:free_riding_control}), to send to the more reliable peers
a higher number of chunks. In other words, ACS adapts the round-time
of each peer to its capacity, in such a way that the higher the
capacity, the lower the round-time.

ACS should be used if we want that some peers to put the uploading
capacity than others cannot provide. A possible situation could be
when we want to mix the CS and P2P models, sending more chunks from
the contents provider's hosts to the users's peers, with the
objective, for example, of incrementing the QoS. Because monitors
should lost a minimum number of chunks (supposely the hosts of the
contents provider should have enough capacity), the ratio of chunks
emitted by the contents provider's hosts can be controlled by using
ACS, and a high enough number of monitor peers (the more monitors, the
higher the ratio).

Notice that ACS only affects the behavior of the splitter.

\begin{comment}
In most overlay configurations (which should exhibit a full-(or
nearly-full) connected topology), all nodes should transmit exactly
the same number of chunks. This that can be considered a disadvantage
in some situations such as for example, in a ``pay-per-view''
streaming service where the QoS must be guaranteed to all users, even
if they can not send as fast as they receive.

\subsubsection{Uploading capacity estimation}
In full-connected teams, the number of chunks that a peer must
retransmit depends on the number of chunks that the peer receives from
the splitter. Moreover, the splitter can estimate the effective
uploading capacity of the peers by checking the number of times that a
peer has lost a chunk. This estimation can be used to maximizing the
QoS by assigning a different chunk-rate to each peer, depending on its
estimated capacity. In other words, if a peer does not loss chunks,
then its chunk-rate will be increased and viceversa.

\subsubsection{Adaptive chunk-rate (round-robin) scheduling}
By default, the round cycle (the frequency that the splitter sends a
chunk to a peer) of all peers is the same. ACS alters this by using,
for a peer $P_i$, a team cycle proportional to the packet-loss-ratio
of $P_i$. While $P_i$ is lossing chunks, the splitter will send to it
less chunks and will send more chunks to the rest of the team. If the
rest of the team can assume the extra transmission capacity that $P_i$
can not afford, the overlay will be stable.
\end{comment}

\begin{comment}
  All nodes of a team (peers and Splitter) that implements only the DBS
transmitt exactly the same amount of media, which basically imply that
if a peer can not fulfill this requirement it will be thrown off the
team. This could be considered too demanding in some specific
configurations, such as a team of colleagues that want to share a
content regardless of who spends more transmission bandwidth or in PPV
(Pay-Per-View) systems where the stream must be guaranteed to those
users that have paid for receiving the stream.

The number of chunks that ultimately a peer must retransmit depends
exclusively on the number of chunks that the peer receives from the
Splitter. Besides, the Splitter knows the performance of the peers of
the team for the task of retransmitting the chunks by checking the
number of times that the peers has lost a chunk. This knowledgment
could be exploited by the Splitter to maximize the profiting of the
team capacity by assigning a different chunk-rate to the peers
depending on their reliability. In other words, if a peer does not
loss chunks then a higher number of chunks is sent to it than to the
rest of them.

By default, as a consequence of the use of the Round-Robin scheduler
(see Rule~\ref{rul:round-robin}) and supposing a constant bit-rate
encoded media, the Splitter sends a chunk to each peer almost
periodically. Moreover, peers use the reception of a chunk from the
Splitter as a timing event to decide when to switch from the
congestion avoidance control mode (see
Rule~\ref{rul:congestion_avoidance_mode}) to the burst mode (see
Rule~\ref{rul:burst_mode}). Taking into account all this information,
the ACS proposes an adaptive Round-Robing scheduler in which the
sending period of a chunk from the Splitter to a peer $P$ is
proportional to the packet loss ratio of $P$. For example, lets
suppose that peers $P_1$, $P_2$ and $P_3$ belongs to the same team and
that in the last period of time $P_1$ has not lost chunks, $P_2$ has
lost 2 chunks and $P_3$ has lost one chunk. In this situation, the
next 7 chunks will be transmitted following the progression: $P_1,
P_1, P_3, P_1, P_1, P_2, P_3$, if we also suppose that the peers are
stores in the list of peers of the Splitter in order.

\end{comment}
