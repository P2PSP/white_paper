DBS does not imposes any control over the grade of solidarity of the
peers. This means that some selfish peers (on simply peers with
reduced connectivity as a consecuence, for example, of NAT issues) can
stay in the team thanks to the generosity of the rest of peers. If
this is unnaceptable, this set of rules forces to all the peers of the
team to share the same number of chunks that they receive.

To achieve this behavior, FCS defines that, if ${\cal P}^j_k$ realises
that $\mathtt{debt}[{\cal P}^j_l]>\mathtt{debt}_{\text{max}}$, then
${\cal P}^j_k$ removes ${\cal P}^j_l$ from ${\cal T}^j_k$. Obviously,
${\cal P}^j_l$ should churn, unless it not interested in playing the
media.

\begin{comment}
In each round, peers check if a chunk have been received from the rest
of peers of the team (${\cal P}_k\in {\cal T}_j)$). If not, peers send
a $[\mathtt{propagate}~{\cal P}_i]$ to one or more (possibly
to the rest of) peers of the team, where ${\cal P}_i$ is the origin peer
of the missing chunk. At this point, the process continues as
described in Section~\ref{dbs:chunk_flooding}.
\end{comment}

\begin{comment}
For each ${\cal P}_k\in N({\cal P}_i)$, ${\cal P}_i$ checks if a chunk
has been received from ${\cal P}_k$. If ${\cal P}_i$ detects that
${\cal P}_k$ has not sent a chunk to it during $L$ consecutive rounds,
performs $N({\cal P}_i) = N({\cal P}_i)\setminus{\cal P}_k$, and stops
sending to ${\cal P}_k$ more chunks.
\end{comment}
\begin{comment}
computes a
``chunk-debt'', denoted by $d({\cal P}_k)$, that is incremented each
time a chunk is received from ${\cal P}_k$ and decremented each time a
chunk is sent to ${\cal P}_k$. If ${\cal P}_i$ verifies that $d({\cal
  P}_k)>D$ (the maximum debt), then ${\cal P}_i$ considers that ${\cal
  P}_k$ is unable to communicate with it, performs $N({\cal P}_i) =
N({\cal P}_i)\setminus{\cal P}_k$, and stops sending to ${\cal P}_k$
more chunks.
\end{comment}
