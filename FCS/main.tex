%%% Local Variables:
%%% mode: latex
%%% TeX-master: "<none>"
%%% End:

DBS does not imposes any control over the grade of solidarity of the
peers. This means that selfish peers (or simply peers with
reduced connectivity as a consecuence, for example, of NAT issues) can
stay in the team thanks to the generosity of the rest of peers. If
this is unnaceptable, this set of rules try to that all the peers of the
team to share the same number of chunks that they receive.

To know the level of solidarity between neighbor peers, each peer
implements a table of chunk debts, $\text{debt}[]$. Every time $P_k$
sends a chunk to $P_l$, $P_k$ runs $\text{debt}[P_l] =
\text{debt}[P_l]+1$, and $P_l$ runs $\text{debt}[P_k] =
\text{debt}[P_k]/2$. So, peers increment the insolidarity measurement
incrementally and decreases it exponentially towards zero. This
respond to the idea of that, if a peer keeps sending chunks to their
neighbors, it has doing its best in retransmitting the chunks it must
forward.

With the $\text{debt}[]$ information, peers modify the way the
neighbor are selected during the flooding procedure. Now, the
$\text{pending}[]$ list is sorted by debts (those peers with a lower
debt will appear in the first positions of $\text{pending}[]$), and
when a peer receives a chunk from the splitter, the run over the
$\text{pending}[]$ list will be reset. In this way, supportive peers
will be served first, incrementing the QoE of the corresponding
users. On the other hand, those peers with a higher chunk debt will
tend to be unserved if no enough bandwidth is available.

The second mechanism to increase the grade of solidarity is to send
the $[\mathtt{request} \text{lost\_chunk\_number}]$ to those peers
with a higher debt. So, if the insolidarity is produced by a overlay
topology imbalance, badly connected peers peers can mitigate this
problem forwarding more chunks to their neighbors.

\begin{comment}
To achieve this behavior, FCS defines that, if ${\cal P}^j_k$ realises
that $\mathtt{debt}[{\cal P}^j_l]>\mathtt{debt}_{\text{max}}$, then
${\cal P}^j_k$ removes ${\cal P}^j_l$ from ${\cal T}^j_k$. Obviously,
${\cal P}^j_l$ should churn, unless it not interested in playing the
media.

, where those peers with
a low chunk debt are selected first.  Debs are clipped to $\pm
D$. In ideal circunstances, debs should be $0$. Obviously, a high
supportivity means a low debt, and viceversa.

In DBS, the splitter sends to each peer of the team one chunk per
round. On the other hand, the peers can have a variable number of
neighbors. In this context, those peers with a higher degree of
connectivity will forward more chunks than the peers with a lower
degree. So, by definition, peers with a lower connectivity will
forward a lower number of chunks for those peers that it is the origin
peer. In the extreme case, a peer $P_x$ behind a NAT could be
connected only with one external peer $P_y$ which should forward to it
all the chunks except those that receive directly from the
splitter. Obviously, in this case, $\text{debt}[P_x]$ in $P_y$ will
reach $D$ fastly.

Peers will remove as neighbors those peers whose debt reaches $D$
during $D^*$ consecutive rounds.
\end{comment}

\begin{notex}
  The prioritized round-robin neighbor selection has not yet been
  implemented as it has been explained here. The $\text{debt}[]$
  structure exists, but is used for a different purporse.
\end{notex}


\begin{comment}
In each round, peers check if a chunk have been received from the rest
of peers of the team (${\cal P}_k\in {\cal T}_j)$). If not, peers send
a $[\mathtt{propagate}~{\cal P}_i]$ to one or more (possibly
to the rest of) peers of the team, where ${\cal P}_i$ is the origin peer
of the missing chunk. At this point, the process continues as
described in Section~\ref{dbs:chunk_flooding}.
\end{comment}

\begin{comment}
For each ${\cal P}_k\in N({\cal P}_i)$, ${\cal P}_i$ checks if a chunk
has been received from ${\cal P}_k$. If ${\cal P}_i$ detects that
${\cal P}_k$ has not sent a chunk to it during $L$ consecutive rounds,
performs $N({\cal P}_i) = N({\cal P}_i)\setminus{\cal P}_k$, and stops
sending to ${\cal P}_k$ more chunks.
\end{comment}
\begin{comment}
computes a
``chunk-debt'', denoted by $d({\cal P}_k)$, that is incremented each
time a chunk is received from ${\cal P}_k$ and decremented each time a
chunk is sent to ${\cal P}_k$. If ${\cal P}_i$ verifies that $d({\cal
  P}_k)>D$ (the maximum debt), then ${\cal P}_i$ considers that ${\cal
  P}_k$ is unable to communicate with it, performs $N({\cal P}_i) =
N({\cal P}_i)\setminus{\cal P}_k$, and stops sending to ${\cal P}_k$
more chunks.
\end{comment}
