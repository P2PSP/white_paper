When peers receive chunks, flood them to the rest of peers of the team
(except to the sender), using again a round-robin schema (see
Fig.~\ref{fig:chunk_generation_and_flooding}). For each duplicate
chunk received by a ${\cal P}_k \in {\cal T}_j$ from ${\cal P}_l \in
{\cal T}_j$, ${\cal P}_k$ sends to ${\cal P}_l$ a
$[\mathtt{prune}~{\cal P}_i]$, where ${\cal P}_i$ is the origin peer
of the duplicate chunk. Thus, only the first ${\cal P}_l$ to send to
${\cal P}_k$ a chunk ``originated'' at ${\cal P}_i$ will do that in
the future, at least that ${\cal P}_k$ revokes this routing
information by sending a $[\mathtt{not~prune}~{\cal P}_i]$ to one or
more (possibly the rest of) peers of the team.

%When peers receive chunks from their splitter, they must flood them to
%their neighbors until the chunks are broadcasted to the whole team
%(Fig.~\ref{fig:chunk_generation_and_flooding}). Lets suppose that
%${\cal P}_k$ receives a chunk. In the case the sender is its splitter,
%${\cal P}_k$ floods the chunk to $N({\cal P}_k)$. However, if the
%sender is a peer ${\cal P}_m\in N({\cal P}_k)$, ${\cal P}_k$ adds
%${\cal P}_m$ to $N({\cal P}_k)$ if ${\cal P}_m$ is a new neighbor, and
%forwards the chunk to the rest of its neighborhood ${\cal P}_n\in
%N({\cal P}_k)\setminus{\cal P}_m$ if ${\cal P}_k$ is in the shortest
%between ${\cal P}_n$ and the origin peer ${\cal P}_i$ of the relayed
%chunk. This will be true if ${\cal P}_k$ is the gateway of ${\cal
%  P}_n$ to go from ${\cal P}_n$ to ${\cal P}_i$. Therefore, a flooding
%with prunning based on shortest path routing is used.
