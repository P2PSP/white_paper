\begin{table}
  \begin{tabular}{rl}
    Limits   & Meaning \\
    \hline
    $G$     & Maximum number of teams in a channel \\
    $N$     & Maximum number of peers in a team \\
    ~\\
    Entity  & Role \\
    \hline
    $O$     & the media source \\
    $T$     & all the teams related to the same media channel \\
    $R$     & teams tracker (relates channels with teams)\\
    $S$     & the set of splitters broadcasting the same channel \\
    $T^{t\in\{1,\cdots,|T|\}}$ & the $t$-th team \\
    $S^{t\in\{1,\cdots,|T|\}}$ & the splitter of the $t$-th team \\
    $P^t_i$ & $i$-th peer of team $T^t$ \\
    ~\\
    Variable & Meaning \\
    \hline
    $1\leq |T|\leq G$   & current number of teams broadcasting the same channel \\
    $1\leq |S|\leq G$ & current number of splitters related with $T$ \\
    $1\leq |T^t|\leq N$ & current number of peers in $T^t$ \\
    $B$     & buffer size in chunks \\
  \end{tabular}
  \caption{Nomenclature used in this
    documentation.\label{tab:nomenclature}}
\end{table}

The nomenclature used for describing P2PSP is shown in
Table~\ref{tab:nomenclature}. In this documentation, only one media
channel has been considered, which basically means that all these
objects should be instantiated for each different channel. In P2PSP, a
channel is the equivalent to a same concept, for example, in the DVB
(Digital Video Broadcasting) system.
