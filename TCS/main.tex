% Emacs, this is -*-latex-*-

\begin{notex}
  Unfinished.
\end{notex}

\subsection{Team Clustering Set}
In some context, especially in flash-crowd situations, it can be
interesting to have several teams broadcasting the same content. Two
teams use the same content if it is verified that 

If $f=1$, in order to limit the latency, the size of the teams must
have a limit $N_\text{max}$ and the content provider should deploy so
many teams as neccesary. All teams should be syncrhonized.

\subsubsection{Tracker redirection}
When an arriving peer wants to join a team, it should contact with a
\emph{tracker} which will redirect it to a team (in fact, to a
splitter) satisfiying that $N<N_\text{max}$.

\subsubsection{Moving peer between teams}
As it will be explained in Section~\ref{sec:latency_analysis}, it can
be neccesary to move already incorporated peers between teams. To
provide this functionality, the tracker sends to the origin splitter a
message [move, peer, team]. The origin splitter sends to the peer a
message [move, team]. Then, the moving peer joins the destination team
and when all chunks of the buffer has been recived twice, the peer
removes the neighbors that belongs to the origin team of its list of
peers.

%%

Obviously, as a consequence of the churn, teams are dynamic. This
means that peer can leave a team and join a different one, by their
own choice. However, in some specific situations can be desirable to
move peers from a team to other. For example, after a massive peers
departure from several teams, it could be interesting to gather the
remaining peers in only one team keeping a seamless playback
(maintaining constant the latency). In a different situation, we would
like to increase the number of teams, in order to provide a better
QoS.

\subsection{Multisplitter teams}

\subsection{Team fission}
A team can be divided into two (or more) teams, dividing the list of
peers of the original splitter into two (or more) parts and using each
of them in a different splitter. After that, peers will start
receiving duplicated chunks. Due to peers forget those neighbors that
do not send to them useful chunks (and a duplicated chunk is not
useful). After a certain amount of time, that depends on the
chunk-ratio and the constant $D_\text{max}$, those peers that receive
chunks from the same splitter will forget the peers of the other(s)
team(s).

\subsection{Teams fussion}
The main objective of merging teams is to reduce the bandwidth
consumption at the provider side.  Two (or more) teams, served by
their respective splitters can be merged only if each one of the peers
that we want to move leaves the origin team and join the destination
one, using the procedure described in
Sections~\ref{sec:peer_departure}
and~\ref{sec:peer_joining}. Basically, the splitter of the origin team
stop sending chunk to their peers, which should contact the tracker,
which always should redirect them to the same destination peer.

However, there is an alternative techinque much more respectful with
all peers of all the merged teams. Lets suppose that we have $S$ teams
and that the content provider wants to reduce the bandwidth
consumption to $S/2$. This can be achieved by 
