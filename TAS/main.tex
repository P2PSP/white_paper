%%% Local Variables:
%%% mode: latex
%%% TeX-master: "<none>"
%%% End:

\label{sec:TAS}

In TAS, the splitter request to each peer of the team the list of
neighbors (peers that send chunks directly, in one hop). This
communication is reliable (TCP) and transmits the lists as a
collection of end-points. The number of requests per round is limited
by the available bandwidth in the overlay, and by the request-ratio
defined at the splitter. Obviously, the higher the ratio, a more
accurate description of the real connectivity in the overlay will be
obtained.

% all peers trace the chunks, appending in the packet the
% end-point of each hup in the overlay network. When at least one of the
% monitors (selected by the splitter) receives each chunk, it sends to
% the splitter the trace for each received chunk. The splitter gathers
% this information and counts the number of times each peer is
% referenced in the traces, in a round. The counter for a peer speaks of
% its connectivity degree. Notice that the connectivity graph of the
% overlay can be computed, simply by following the traces.

After knowing the connectivity degree of each peer, the slitter can
adapt the round-robin scheduling of the origin peers by sending a
number of chunks proportional to the inverse of the degree of the
origin peer.

