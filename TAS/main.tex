In TAS, all peers trace the chunks, appending in the packet the
end-point of each hup in the overlay network. When at least one of the
monitors (selected by the splitter) receives each chunk, it sends to
the splitter the trace for each received chunk. The splitter gathers
this information and counts the number of times each peer is
referenced in the traces, in a round. The counter for a peer speaks of
its connectivity degree. Notice that the connectivity graph of the
overlay can be computed, simply by following the traces.

After knowing the connectivity degree of each peer, the slitter can
adapt the round-robing scheduling of the origin peers by sending a
number of chunks proportional to the inverse of the degree.
