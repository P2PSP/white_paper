\title{The LOCO-I Standard (ISO/IEC 14495-1)}

\maketitle
\tableofcontents

\section*{Intro}
\begin{itemize}
\item LOCO-I~\cite{weinberger96} ({\sl LOw COmplexity, context-based,
    lossless Image compression algorithm}), also named JPEG-LS (Joint
  Photographic Expert Group - LosslesS), is a lossless stil image
  codec based on context-based predictive coding and Golomb Coding.
\item Developed by HP, that only offers the executable code in:
  \texttt{http://www.hpl.hp.com/loco/locodown.htm}
\end{itemize}

\section{Encoder}
\begin{enumerate}
\item {\bf Initialization of the prediction contexts}:
\begin{enumerate}
\item Let $Q$ the actual context (there are $1094$ different
  spatial contexts).
\item Let $N[Q]$ the number of times that the context $Q$ has been
found.
\item Let $B[Q]$ the accumulated prediction error for the
  context $Q$.
\item Let $A[Q$ the sumatory of the absolute value of the
  prediction residues for the context $Q$.
\item Let $C[Q]$ the the bias cancellation values for the
  context ${Q}$. These values are added to the predictions to try
  that the predictions residues have a zero average. If this is not
  satisfied, the compression ratio in reduced severelly because the
  average of the real Lapace probability distribution of the residues
  and the modeled probability distribution does not match. For this
  reason, $C[Q]$ is proportional to $B[Q]/N[Q]$ that added to the
  predictions, cancel the bias.
  
  % \item Sea $C[Q]$ los valores de cancelaci�n del {\sl bias}. El {\sl
  %     bias} es un valor que sumado a la predicci�n espacial provoca que
  %   su media sea 0. Inicialmente $C[Q]\leftarrow 0~\forall Q$.
\end{enumerate}

\item {\bf Determination of the prediction context} ${\cal Q}$:
  \begin{enumerate}
  \item Compute the local gradient:

    \fig{100}{LOCO-I_contexto_prediccion.png}

    \begin{displaymath}
      \begin{array}{rcl}
        g_1 & \leftarrow & d-a\\
        g_2 & \leftarrow & a-c\\
        g_3 & \leftarrow & c-b\\
        g_4 & \leftarrow & b-e
      \end{array}
    \end{displaymath}
  \item Quantize the gradients:
    \begin{displaymath}
      q_i\leftarrow\left\{
        \begin{array}{ll}
          0 & \text{if~} g_i=0\\
          1 & \text{if~} 1\le |g_i|\le 2\\
          2 & \text{if~} 3\le |g_i|\le 6\\
          3 & \text{if~} 7\le |g_i|\le 14\\
          4 & \text{otherwise}
        \end{array}
      \right.
    \end{displaymath}
    for $i=1,2,3$ and
    \begin{displaymath}
      q_4\leftarrow\left\{
        \begin{array}{ll}
          0 & \text{if~} |g_4|<0\\
          1 & \text{if~} 5\le g_4\\
          2 & \text{otherwise}.
        \end{array}
      \right.
    \end{displaymath}
  \end{enumerate}
  
\item {\bf Compute the residue} $M(e)$:
  \begin{enumerate}
  \item The initial prediction:
    \begin{displaymath}
      \hat{s}\leftarrow\left\{
        \begin{array}{ll}
          \text{min}(a,b) & \text{if~} c\geq\text{max}(a,b) \\
          \text{max}(a,b) & \text{if~} c\leq\text{min}(a,b) \\
          a+b-c             & \text{otherwise}.
        \end{array}
      \right.
    \end{displaymath}
    
  \item {\sl Bias} cancellation:
    \begin{displaymath}
      \hat{s}\leftarrow\left\{
        \begin{array}{ll}
          \hat{s}+C[Q] & \text{if~} g_1>0 \\
          \hat{s}-C[Q] & \text{otherwise}.
        \end{array}
      \right.
    \end{displaymath}
    
  \item Compute the prediction error:
    \begin{displaymath}
      e\leftarrow(s-\hat{s}) \text{~mod~}\beta,
    \end{displaymath}
    where $\beta$ is the number of bits/pixel. This produces a
    projection of the residues from the dynamic range
    $[-\alpha+1,\alpha-1]$ to $[-\alpha/2,\alpha/2-1]$, where
    $\alpha=2^\beta$ is the size of the source alphabet.
    
  \item Shuffle the residues in order to get an exponiential (with
    negative exponent) distribution of the probability of the
    residues:
    \begin{displaymath}
      M(e)\leftarrow\left\{
        \begin{array}{ll}
          2e & \text{if~} e\ge 0\\
          2|e|-1 & \text{otherwise}.
        \end{array}
      \right.
    \end{displaymath}
    After that, the residues are in the range $[0, 1, \cdots, 2^\beta-1]$
  \end{enumerate}
  
\item {\bf Variable length encoding of $M(e)$ in the context $Q$}:
  \begin{enumerate}
  \item Output $\leftarrow$ a Rice code for $M(e)$ using the slope
    $k=\lceil\log_2(A[Q])\rceil$.
  \end{enumerate}
  
\item {\bf Update the context} $Q$:
  \begin{enumerate}
  \item $B[Q] \leftarrow B[Q] + e$.
  \item $A[Q] \leftarrow A[Q] + |e|$.
  \item If $N[Q]=\mathsf{RESET}$, then: (where
    $64\le\mathsf{RESET}\le 256$)
    \begin{enumerate}
    \item $A[Q] \leftarrow A[Q]/2$.
    \item $B[Q] \leftarrow B[Q]/2$.
    \item $H[Q] \leftarrow N[Q]/2$.
    \end{enumerate}
  \item $N[Q] \leftarrow N[Q]+1$.
  \item {\bf Update of $C[Q]$}: 
    \begin{enumerate}
    \item If $B[A]\le -N[Q]$, then:
      \begin{enumerate}
      \item $B[Q] \leftarrow B[Q]+N[Q]$.
      \item If $C[Q]>-128$, then:
        \begin{itemize}
        \item $C[Q] \leftarrow C[Q]-1$.
        \end{itemize}
      \item If $B[Q]\le -N[Q]$, then:
        \begin{itemize}
        \item $B[Q] \leftarrow -N[Q]+1$.
        \end{itemize}
      \end{enumerate}
    \item Else:
      \begin{enumerate}
      \item If $B[Q]>0$, then:
        \begin{itemize}
        \item $B[Q] \leftarrow B[Q]-N[Q]$.
        \item If $C[Q]<127$, then:
          \begin{itemize}
          \item[$\star$] $C[Q] \leftarrow C[Q]+1$.
          \end{itemize}
        \item If $B[Q]>0$, then:
          \begin{itemize}
          \item[$\star$] $B[Q] \leftarrow 0$.
          \end{itemize}
        \end{itemize}
      \end{enumerate}
    \end{enumerate}
  \end{enumerate}
\end{enumerate}

\section{Decoder}
\begin{enumerate}
\item Idem to Step 1 of the encoder.
\item Idem to Step 2 of the encoder.
\item Decode $M(e)$.
\item Compute the initial prediction such as in Step 3.a of the
  encoder.
\item Add the bias to $\hat{s}$:
  \begin{displaymath}
    \hat{s}\leftarrow\left\{
      \begin{array}{ll}
        \hat{s}-C[Q] & \text{if~} g_1>0 \\
        \hat{s}+C[Q] & \text{otherwise}.
      \end{array}
    \right.
  \end{displaymath}
\item Recover the original Laplace distribution for the residues:
  \begin{displaymath}
    e\leftarrow M^{-1}(M(e))=\left\{
      \begin{array}{ll}
        -M(e)/2-1 & \text{si~} M(e) \text{is~odd}\\
        M(e)/2 & \text{otherwise}.
      \end{array}
    \right.
  \end{displaymath}
\item $s\leftarrow (e+\hat{s})\text{~mod~}\beta$.
\item Update $Q$ as in Step 5 of the encoder.
\end{enumerate}

\section{The RLE mode}
LOCO-I uses a Rice encoder that can be very redundant if the
probability distributions are very narrow. To overcome this drawback,
there is a special encoding mode for this situation that is triggered
when $a=b=c=d$.

The normal mode is re-started when if $s\ne a$ or the end of a line
has been reached.

\section*{Let's go to the lab!}
\begin{itemize}
\item Download the LOCO-I command line executables from
  \url{http://www.hpl.hp.com/loco/locodown.htm}.
  
\item Fill the table:
\begin{verbatim}
 Codec | lena boats pepers zelda Average
-------+---------------------------------
LOCO-I | 
\end{verbatim}
\end{itemize}

\bibliography{JPEG2000}
