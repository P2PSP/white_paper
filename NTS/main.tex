% NTS

Most peers run behind NATs. All processes behind the same NAT box will
use the same public IP address of the NAT. Basically, there exist two
different types of NATs: (1) cone, and (2) symmetric.

A cone NAT uses the same public end-point for all the private
processes that run behind the NAT. This means that, if two or more
peers run in the same private cone-NATed network, all of them will
share the same public IP address and will use a different public
port. The important thing here is that, if a peer ${\cal P}^j_i$ is
using the end-point $X$, any incomming traffic directed to $X$ will be
forwarded to ${\cal P}^j_i$. The situation is identical to the case in
which all peers run in the same host. In this context, DBS should be
able to establish a (UDP) communication among the peers because when
peers send the $[\mathtt{hello}]$ messages while they are receiving
the list of peers from the splitter, the necessary translation entries
are created in the NAT.

On the other hand, a symmetric NAT uses a different public port for
every connection. Thus, if a (symmetric-NATed) peer ${\cal P}^j_i$ is
assigned the public end-point $X_0$ to communcate with the splitter, a
different public end-point $\{X_1, \cdots, X_M\}$ will be assigned by
the NAT for the rest of nodes of the overlay, when the sequence of
$[\mathtt{hello}]$'s are relayed by the NAT. This behavior can be or
not can be a problem for DBS, depending on if the other peers are not
or are behind a symmetric NAT. In the first case, when only one of the
interlocutors of the $[\mathtt{hello}]$ message is behind a symmetric
NAT, the DBS handshake should work, establishing a UDP channel between
both peers. However, in the second case in which both peers are behind
different symmetric-NATs, the communication will success only if the
$[\mathtt{hello}]$ messages are sent to the public end-point that the
other NAT would use to communicate with the incomming peer, and, at
the same time, the older peer creates the corresponding translation
entry in its NAT. Obviously, this will be possible only if both peers
are able to force or at least predict the public port that their NATs
will use for the $[\mathtt{hello}]$'s. Usually, peers can not
configure their NAT entries. Therefore, only the second option is
available and some kind of port prediction procedures becomes
necessary.
