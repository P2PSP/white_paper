%%% Local Variables:
%%% mode: latex
%%% TeX-master: "<none>"
%%% End:

\label{sec:NTS}

Most of the peers run inside of ``private'' networks, i.e. behind NAT
devices. NTS\footnote{This work was supported by GSoC15, developed by
Max Mertens, and mentored by Vicente González-Ruiz and Juan Pablo
Gracía-Ortiz. The original information is available
at \url{https://github.com/P2PSP/core/tree/master/doc/NTS}.} is an DBS
extension which provides peer connectivity for some NAT configurations
where DBS can not provide direct peer
communication.\footnote{Normally, this problem does not prevent
prevent the peers from receiving the stream. However, a low degree of
connectivity among peers reduces the tolerance of the team to
transmission errors and attacks, and makes that the peers with a
higher degree to send more chunks that the others, which can be
considered undesirable in specific environments.}

Peers behind the same NAT will use the same external (also called
``public'', because in most cases we have not nested NAT
configurations) IP address of the NAT. Basically, there exist two
different types of NATs: (1) \emph{cone}, and (2) \emph{symmetric}. At
the same time, NATs can implement different filtering strategies for
the packets that comes from the external side: (a) \emph{no
  filtering}, (b) \emph{source IP filtering}, and (c) \emph{source
  end-point filtering}. Finally, NATs can use several port allocation
algorithms, among which, the most frequent are: (i) \emph{port
  preservation} and (ii) \emph{random port}. Notice that in this
discussion, only UDP transmissions will be considered.

\subsection{Traffic filtering}
Lets suppose a team in which, for the sake of simplicity, there is
only one external (public) peer $P_e$, and that a new internal
(private) peer $P_i$ has sent the sequence of [$\mathtt{hello}$]'s
(see Sec~\ref{sec:joining}). Lets denote $P_i$'s NAT as $A$. When no
filtering is used at all, $A$ forwards to $P_i$ any external packet
that arrives to it (obviously, if it was sent to the entry in $A$'s
translation table that was created during the transmission of the
sequence of [$\mathtt{hello}$]'s), independently on the source
end-points of the packets. In the case of source (IP) address
filtering, $A$ will forward the packets only if they come from
$P_e$'s host.  When source end-point filtering is used, $A$ also
checks the source port, i.e., that the packets were originated at
$P_e$'s end-point.

\begin{figure}
  \myfig{graphics/cone_NAT}{5cm}{500} %
  \caption{Cone NAT port allocation.} %
  \label{fig:cone_NAT}
\end{figure}

\begin{figure}
  \myfig{graphics/symmetric_NAT}{5cm}{500} %
  \caption{Symmetric NAT port allocation.} %
  \label{fig:cone_NAT}
\end{figure}

\subsection{Cone VS symmetric}
Cone NATs use the same external end-point for every packet that comes
from the same internal end-point, independently on the destination of
the packets (see Fig.~\ref{fig:cone_NAT}). For the external peer
$P_e$, the situation is identical to the case in which the NATed peer
$P_i$ would be running in a public host.

Symmetric NATs use different external end-points for different packets
that comes from the same internal end-point, when these packets have
different destination end-points (see
Fig.~\ref{fig:symmetric_NAT}). Thus, two different external peers will
see two different public end-points of $P_e$.

\subsection{Port allocation}
In the case of port preservation, if $X$:$Y$ is the private end-point
(IP address:port) of a UDP packet, the NAT will use the public port
$Y$, if available (notice that $Y$ cound have been assigned to a
previous communcation). If $Y$ were unavailable, the NAT usually will
assign the closer free port (this is called ``sequentially port
allocation''), usually by increasing the port value, although this
behavior has not been standarized at all.

When random port allocation is implemented, the public port will be
assigned at random. Notice that, even in SN-PPA configurations, in
most of the real situations (where peers must compete with the rest of
processes that use the network for the same NAT resources,) some kind
of randomization should be always expected during a the port
assignment.

\subsection{NAT type analysis}
An incoming peer $P_i$ can determine its NAT behavior using the
following steps:
\begin{enumerate}
\item Let $A_0, A_1, \cdots, A_M\}$ the public ports used by peer
  $P_i$, whose NAT is $A$, to send the [$\mathtt{hello}$] UDP packets
  towards the splitter $S$ and the $M$ monitor peers of the team, in
  this order. This data is known by $P_i$ after receiving the
  acknowledgment of each [$\mathtt{hello}$]. Compute
  \begin{equation}
    \Delta_k = \mathsf{A}_k - \mathsf{A}_{k-1}
    \label{eq:port_distancies}
  \end{equation}
  for $k=1,2,\cdots,M$, the \emph{port distances} gathered by $P_i$.
\item Determine a \emph{port step}
  \begin{equation}
    \Delta = \left\{\begin{array}{lr}
    0, & \text{if } \forall i, \Delta_i = 0 \\
    \mathrm{GCD}(\Delta_1, \cdots, \Delta_m), & \text{otherwise}
    \end{array}\right.
    \label{eq:port_step}
  \end{equation}
  where GCD is the Greatest Common Divisor operator.
\item If $\Delta=0$ ($A$ is using the same external port for
  communicating $P_i$ with the rest of peers of the team) then $P_i$
  is behind a cone NAT. Notice that public (not NATed) peers will be
  considered as being using this type of NAT, also.
\item If $\Delta>0$ ($A$ is using a different external port for each
  external peer) then $P_i$ is behind a symmetric NAT. In this case:
  \begin{enumerate}
  \item If
    \begin{equation}
      \Delta_1 = \Delta_2 = \cdots = \Delta_m
    \end{equation}
    then $A$ is using sequentially port allocation.
  \item If
    \begin{equation}
      \Delta = \lim_{m\to\infty} \mathrm{GCD}(\Delta_1, \cdots, \Delta_m) = 1.
    \end{equation}
    then $A$ is using random port allocation.
  \end{enumerate}
\end{enumerate}

\subsection{(Theoretical) NAT traversal performance of DBS}

\begin{table}
  \centering
  \begin{tabular}{l|r|r|r|r|r}
    Peer1/2 & CN    & CN-AF & CN-EF & SN-PPA & SN-RPA \\
    \hline
    CN      & DBS   & DBS   & DBS   & DBS    & DBS    \\
    CN-AF   & DBS   & DBS   & DBS   & NTS    & -      \\
    CN-EF   & DBS   & DBS   & DBS   & NTS    & -      \\
    SN-PPA  & DBS   & NTS   & NTS   & NTS    & -      \\
    SN-RPA  & DBS   & -     & -     & -      & -
  \end{tabular}
  \caption{NAT traversal success for different NAT typical
    combinations. CN-NF (also known by ``full cone NAT'') stands for
    Cone NAT (without packet filtering). CN-AF (also known as
    ``restricted cone NAT'') stands for Cone NAT with source Address
    Filtering. CN-EF (also known by ``port restricted cone NAT'')
    stands for Cone NAT source End-point Filtering. SN-PPA stands for
    Symmetric NAT Port Preservation Allocation, and no packet
    filtering has been considered. SN-RPA stands for Symmetric NAT
    Random Port Allocation, and no packet filtering has been used.} %
  \label{tab:theoretical}
\end{table}

\begin{figure}
  \myfig{graphics/UDP_Hole_Punching_RCN}{6cm}{600} %
  \caption{An example that shows how it is possible to establish a
    connection with DBS when two peers $P_1$ and $P_2$ that are behind
    cone NATs.} %
  \label{fig:UDP_Hole_Punching_RCN}
\end{figure}

\begin{figure}
  \myfig{graphics/UDP_Hole_Punching_SN_failure}{6cm}{600} %
  \caption{An example that shows why its is impossible to establish a
    connection with DBS when two peers $P_1$ and $P_2$ that are behind
    symmetric NATs.} %
  \label{fig:UDP_Hole_Punching_SN_failure}
\end{figure}

\begin{figure*}
  \myfig{graphics/SNTuCPP}{8cm}{800} %
  \caption{Timeline of an (ideal) NTS interaction between two peers
    $P_1$ and $P_2$ which are behind symmetric NATs.} %
  \label{fig:SNTuCPP}
\end{figure*}

Table~\ref{tab:theoretical} shows the theoretical traversal success of
DBS (or an extension of it) for different NAT type combinations. Peer1
represents to a peer already joined to the team, and Peer2 to an
incoming peer. The entries labeled with ``DBS'' are those that will be
handled by DBS, out-of-the-box. An explanation of why the DBS
handshake works for such configurations is shown in
Fig.~\ref{fig:UDP_Hole_Punching_RCN}.  Notice that source end-point
filtering has been used in this example, although a similar results
should be obtained for simple source address filtering. On the other
hand, the combinations labeled with ``-'' or ``NTS'' will not work
with DBS (see Fig.\ref{fig:UDP_Hole_Punching_SN_failure}). In fact,
only the ``NTS'' entries should work, in general, with NTS, depending
on the port prediction algorithm and the number of
tries.

Fig.~\ref{fig:SNTuCPP} shows an example of an NTS (NAT traversal)
success. When the new NATed peers, $P_1$ and $P_2$, arrive at the
team,
%\footnote{The same principle applies if
%  there is more than one monitor or more than one already incorporated
%  peer in the team.}
the following events happen:

\begin{itemize}
  \item [00.] $M$ requests to join the team (the joining request is
    not shown in the figure for brevity) and $S$ sends to $M$ an empty
    list of peers. At this moment, $M$ has joined the team.
  \item [01.] $P_1$ requests $S$ to join through an external port
    $A_0$ (again, this message is not shown).  $S$ sends to $P_1$ the
    list of peers. This list contains only the endpoint of $M$.
  \item [02.] NAT $A$ relays towards $P_1$ the previuos message.
  \item [03.] $P_1$ answers $[\mathtt{hello} M]$ to $M$.
  \item [04.] $A$ relays the previous message, which is received by
    $M$. Due to $A$ is a symmetric NAT, a new source port $A_1$ is
    used for this message.
  \item [05.] $M$ sends $[\mathtt{ack} M]$ towards $(A, A_1)$.
  \item [06.] The previous message is relayed by $A$. Simultaneously,
    $M$ informs to $S$ that $P_1$ has communicated
    with it, using the external endpoint $(A,A_1)$.
  \item [07.] $S$ acknowledges the reception of the
    previous message.
  \item [08.] $P_2$ requests to join the team (not shown) and $S$
    sends to it the current list of peers, which contains the endpoint
    of $M$ and the tuple $((A,A_0),\Delta_{A},\#P_2)$ (the external
    endpoint used by $P_1$ to communicate with $S$, the maximum port
    step in NAT $A$, $\Delta_{A}$ measured by $S$ for $P_1$
    thoroughtout its incorporation to the team, and the index of
    $P_2$, $\#P_2$, in the list of peers). Using this information,
    $P_2$ will perform the port prediction for the external port that
    $\cal{A}$ will assign to $P_1$ when it be communicating with
    $P_2$. This prediction is the list of ports
    $Z_{\#P_1}=${~\tt{\footnotesize
        get\_guessed\_ports}($\#P_2${\footnotesize,}$A_0${\footnotesize,}$\Delta_A$)}
    is populated by $P_2$ using the Algorithm~\ref{alg:prediction}.
  \item [09.] $B$ retransmits the previous message.
  \item [10.] $P_2$ sends a $[\mathtt{hello} M]$ towards $M$.
  \item [11.] $B$ retransmits the previous message, which
    arrives to $M$, and $P_2$ sends a
    $[\mathtt{hello} (A,A_2)]$ towards
    $(A,A_2)$, which has been computed in the Step 08.
  \item [12.] The previous message arrives to $(A,A_2)$
    (which is correct), but $A$ discards this packet
    because still there is not a working entry in its translation
    table for the key
    $((B,B_2),A_2)$.
  \item [13.] $M$ acknowledges the $[\mathtt{hello} M]$, which arrived
    in the Step 11.
  \item [14.] The $[\mathtt{ack} M]$ message is received by $P_2$ and
    $M$ informs to $S$ that $P_2$ is also using the port $B_1$ (this
    information is used to compute the maximum port step $\Delta_B$
      in NAT $B$, measured for $P_2$ thoroughout its incorporation.
  \item [15.] $S$ acknowledges the previous reception.
  \item [16.] $S$ sends to $P_1$ the message $[(B,B_0),\Delta_B,S']$
    (external end-point used by $P_2$ to talk with $S$, port step
    measured for $P_2$ and a new temporaly listenning port
    $S'$ at node $S$). The tuple
    $((B,B_0),\Delta_B)$ allows $P_1$ to
    predict which external port ($B_2$) will use
    $\mathrm{NAT} B$ when $P_2$ sends a packet to $P_1$. The
    extra socket bound by $S$ to $S'$ will be used to update
    the external port that $P_1$ is currently using to communicate
    with the rest of peers of the team. % (see Appendix~\ref{ape:SYMPP}).
  \item [17.] $P_1$ receives the previous message.
  \item [18.] $P_1$ says $[\mathtt{hello} P_2]$ to EEP
    $(\mathrm{NAT} B,B_2)$.
  \item [19.] $P_1$ says $[\mathtt{hello} S]$ to EEP $(S,S')$.
  \item [20.] $\mathrm{NAT} B$ relays the message
    $[\mathtt{hello} P_2]$ towards $P_2$ and $[\mathtt{hello} S]$ is
      received by $S$ (which updates the external port for
      $P_1$). Notice that at this moment, $P_2$ knows that $P_1$ is
      able to send data to it.
  \item [21.] Both, $S$ and $P_2$ acknowledges the $[\mathtt{hello}]$
    messages.
  \item [22.] $[\mathtt{ack} S]$ is received by $P_1$,
    $[\mathtt{ack} P_2]$ is received by $\mathrm{NAT} A$ and the timer
    assigned to the message $[\mathtt{hello} P_1]$ sent in Step 11
    timeouts and this message is re-sent.
  \item [23.] $P_1$ receives $[\mathtt{ack} P_2]$ and $\mathrm{NAT} A$
    receives $[\mathtt{hello} P_1]$.
  \item [24.] $[\mathtt{hello} P_1]$ is delivered to $P_1$. At this
    moment, $P_1$ knows that $P_2$ is able to send data to it.
  \item [25.] $P_1$ acknowledges the previous $[\mathtt{hello} P_1]$.
  \item [26.] $[\mathtt{ack} P_1]$ arrives to $\mathrm{NAT} B$.
  \item [27.] $[\mathtt{ack} P_1]$ arrives to $P_2$.
  \item [28.] $P_1$ and $P_2$ annouce to the $S$ the source port used
    by the other peer.
  \item [29.] This information is received by the $S$, which updates
    the external port information for $P_1$ and $P_2$.
\end{itemize}

Summarizing, NTS can provide connectivity for those peers that are
behind port-preservation symmetric NATs with sequential port
allocation.

\subsection{A port prediction algorithm (Max's proposal)}
When both peers, Peer1 and Peer2, are behind symmetric NATs, both must
predict the port that the NAT of the interlocutor peer will use to
send the packets towards it. And obviously, this must be performed by
each already incorporated peer that is behind a symmetric NAT.

The list of predicted ports $Z$ that a a peer $P_x$ performs is
determined by:
\begin{equation}
  \begin{array}{rcl}
    Z & = & A_0 + x + \{s\in\{0,1,\cdots,N/2-1\}\}; \\
    Z & += & A_0 + (x + \{s\in\{0,1,\cdots, N-1\}\}) \cdot \Delta.
  \end{array}
\end{equation}
where ``$+=$'' denotes the concatenation of lists and $N$ is the
number of guessed ports, $A_0$ is the first external port (the
port used to communicate with $S$) assigned to the incoming peer and
$\Delta$ is the (maximum) port step measured for the incoming peer's
NAT.
