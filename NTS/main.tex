% NTS

Most peers run inside of ``private'' networks, behind NAT boxes. In
this case, all peers behind the same NAT will use the same external
(also called ``public'', because in most cases we have not anidated
confgurations) IP address of the NAT. Basically, there exist two
different types of NATs: (1) \emph{cone}, and (2) \emph{symmetric}. At
the same time, NATs can implement different filtering strategies for
the packets that comes from the external side: (a) \emph{no
  filtering}, (b) \emph{source IP filtering}, and (c) \emph{source
  end-point filtering}. Finally, NATs can use several port allocation
algorithms, among which, the most frequent are: (i) \emph{port
  preservation}, (ii) \emph{random port}. Notice that in this
discussion, only UDP transmission will be considered.

\subsection{Traffic filtering}
Lets suppose a team in which, for the sake of simplicity, there is
only one external (public) peer ${\cal P}^t_e$, and that a new
internal (private) peer ${\cal P}^t_i$ has sent the sequence of
[$\mathtt{hello}$]'s (see Sec~\ref{sec:peer_joining}. Lets denote
${\cal P}^t_i$'s NAT as $\mathsf{A}$. When no filtering is used at
all, $\mathsf{A}$ forwards to ${\cal P}^t_i$ any external packet
that arrives to it (obviously, if it was sent to the only entry in
$\mathsf{A}$'s translation table that was created during the
transmission of the sequence of [$\mathtt{hello}$]'s), independently
on the source end-points of the packets. In the case of source IP
filtering, $\mathsf{A}$ will forward the packets only if they come
from ${\cal P}^t_e$'s host.  When source end-point filtering is used,
$\mathsf{A}$ also checks the source port, i.e., that the packets
were originated at ${\cal P}^t_e$'s end-point.

\subsection{Cone VS symmetric}
Cone NATs use the same external end-point for every packet that comes
from the same internal end-point, independently on the destination of
the packets (see Fig.~\ref{fig:cone}). For the external peer ${\cal
  P}^t_e$, the situation is identical to the case in which the NATed
peer ${\cal P}^t_i$ would be running in a public host.

Symmetric NATs use different external end-points for different packets
that comes from the same internal end-point, when these packets have
different destination end-points (see Fig.~\ref{fig:symmetric}). Thus,
two different external peers will see two different public end-points
of ${\cal P}^t_e$.

\subsection{Port allocation}
In the case of port preservation, if $X$:$Y$ is the private end-point
(IP address:port) of a UDP packet, the NAT will use the public port
$Y$, if available (notice that $Y$ cound have been assigned to a
previous communcation). If $Y$ were unavailable, the NAT usually will
assign the closer free port (this is called ``sequentially port
allocation''), usually incrementing the port value, although this has
not been standarized at all.

When random port allocation is implemented, the public port will be
assigned at random.

\subsection{NAT type analysis}
A incoming peer ${\cal P}^t_i$ can determine its NAT behavior using
the following procedure:
\begin{enumerate}
\item Let $\{\mathsf{A}_0, \mathsf{A}_1, \cdots, \mathsf{A}_M\}$ the
  public ports used by peer ${\cal P}^t_i$, whose NAT is $\mathsf{A}$,
  to send the [$\mathtt{hello}$] UDP packets towards ${\cal S}^t$ and
  the $M$ monitor peers of the team, in this order. This data is known
  by ${\cal P}^t_i$ after receiving the acknowledgment of each
  [$\mathtt{hello}$]. Compute
  \begin{equation}
    \Delta_k = \mathsf{A}_k - \mathsf{A}_{k-1}
    \label{eq:port_distancies}
  \end{equation}
  for $k=1,2,\cdots,M$, the \emph{port distances} gathered by ${\cal
    P}^t_i$.
\item Determine a \emph{port step}
  \begin{equation}
    \Delta = \left\{\begin{array}{lr}
    0, & \text{if } \forall i, \Delta_i = 0 \\
    \mathrm{GCD}(\Delta_1, \cdots, \Delta_m), & \text{otherwise}
    \end{array}\right.
    \label{eq:port_step}
  \end{equation}
  where GCD is the Greatest Common Divisor operator.
\item If $\Delta=0$ ($\mathsf{A}$ is using the same external port for
  communicating ${\cal P}^t_i$ with the rest of peers of the team)
  then ${\cal P}^t_i$ is behind a cone NAT. Notice that public (not
  NATed) peers will be considered as being using this type of NAT,
  also.
\item If $\Delta>0$ ($\mathsf{A}$ is using a different external port
  for each external peer) then ${\cal P}^t_i$ is behind a symmetric
  NAT. In this case:
  \begin{enumerate}
  \item If
    \begin{equation}
      \Delta_1 = \Delta_2 = \cdots = \Delta_m
    \end{equation}
    then $\mathsf{A}$ is using sequentially port allocation.
  \item If
    \begin{equation}
      \Delta = \lim_{m\to\infty} \mathrm{GCD}(\Delta_1, \cdots, \Delta_m) = 1.
    \end{equation}
    then $\mathsf{A}$ is using random port allocation.
  \end{enumerate}
\end{enumerate}

At this moment, we will divide our discussion about the performance of
DBS (or an extension of it) in three different scenarios that a
incoming peer ${\cal P}^t_i$ can found when joining a P2PSP team
${\cal T}^t$: (1) that ${\cal P}^t_i$ is behind a cone NAT, (2) that
${\cal P}^t_i$ is behind a symmetric NAT which uses port preservation
and sequentially port allocation, and (3) that ${\cal P}^t_i$ is
behind a random port symmetric NAT.

\subsection{DBS over cone NATs}
DBS should be able to establish an (UDP) communication among
cone-NATed peers, independently on the packet filtering and port
allocation algorithms that are implemented. An example of this
behavior is shown in Fig.~\ref{fig:UDP-Hole-Punching-RCN}. Notice that
source end-point filtering has been used in this example.

\subsection{DBS over symmetric NATs}


Lets suppose now, that a new private peer ${\cal P}^t_j$ joins the
team through a symmetric NAT $\mathtt{B}$ (see
Fig.~\ref{fig:NATing_context}). In this case, $\mathtt{B}$ assigns
to ${\cal P}^t_j$ two different ports (and therefore, different
public end-points) $\mathsf{B}_e$ (for ${\cal P}^t_e$) and
$\mathsf{B}_i$ (for ${\cal P}^t_i$).

Notice that only one peer (${\cal P}^t_j$) is behind a symmetric NAT

if one of the peers is
behind a symmetric NAT (${\cal P}^j_i$ in our example), the DBS
handshake should also work.

Lets suppose finally, that $\mathtt{A}_j$ is also a symmetric NAT.

 However, when both peers (${\cal P}^j_l$ and
${\cal P}^j_k$, for example) run behind (different) symmetric NATs
(see Fig.\ref{fig:UDP-Hole-Punching_SN_failure}), the port-allocation
algorith dificults (and in some cases, even inhibits) the
communication between the peers. In this situation, the communication
will be success, if and only if, the $[\mathtt{hello}]$ messages are
sent by ${\cal P}^j_i$ (the NATed incoming peer) to the external
end-point that the other NAT would use to communicate with the
incomming peer, and, at the same time, the older peer creates the
corresponding translation entry in its NAT. Obviously, this will be
possible only if both peers are able to force, or at least predict,
the public ports that their NATs will use for the
$[\mathtt{hello}]$'s. Usually, peers can not configure their NAT
entries. Therefore, only the second option is available and some kind
of port prediction procedures becomes necessary.

\subsection{Port allocation}
Port prediction techniques depends on the port allocation algorithm
used by the symmetric NAT. At least 2 different strategies has been
implemented: (1) port-preservation allocation, (2) random port
allocation. In the first case, if $X:Y$ is the private end-point of a
UDP packet, the NAT will use the public port $Y$, if available ($Y$
cound have been assigned to a previous communcation). If $Y$ were
unavailable, the NAT usually will assign the closer free port, usually
in an incremental mode, although this has not been standarized. In the
second case, the public port will be assigned at random.


\subsection{Port prediction}
The list of predicted ports that a the peer $P_x$ performs is
determined by:
\begin{equation}
  \begin{array}{rcl}
    Z_x & = & \textsf{A}_0 + x + \{s\in\{0,1,\cdots,N/2-1\}\}; \\
    Z_x & += & \textsf{A}_0 + (x + \{s\in\{0,1,\cdots, N-1\}\}) \cdot \Delta.
  \end{array}
\end{equation}
where ``$+=$'' denotes the concatenation of lists and $N$ is the
number of guessed ports, $\textsf{A}_0$ is the first external port (the
port used to communicate with $S$) assigned to the incoming peer and
$\Delta$ is the (maximum) port step measured for the incoming peer's
NAT.
