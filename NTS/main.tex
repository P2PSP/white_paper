% NTS

Most most of the peers run inside of ``private'' networks, i.e. behind
NAT devices. NTS\footnote{This work was supported by GSoC15, developed
  by Max Mertens, and mentored by Vicente González-Ruiz and Juan Pablo
  Gracía-Ortiz.} is an DBS extension which provides peer connectivity
for some NAT configurations DBS can not provide direct peer
communication.\footnote{Normally, this problem does not prevent
  prevent the peers from receiving the stream. However, a low degree
  of connectivity among peers reduces the tolerance of the team to
  transmission errors and attacks, and makes that the peers with a
  higher degree to send more chunks that the others, which can be
  considered undesirable in specific environments.}

Peers behind the same NAT will use the same external (also called
``public'', because in most cases we have not nested NAT
configurations) IP address of the NAT. Basically, there exist two
different types of NATs: (1) \emph{cone}, and (2) \emph{symmetric}. At
the same time, NATs can implement different filtering strategies for
the packets that comes from the external side: (a) \emph{no
  filtering}, (b) \emph{source IP filtering}, and (c) \emph{source
  end-point filtering}. Finally, NATs can use several port allocation
algorithms, among which, the most frequent are: (i) \emph{port
  preservation} and (ii) \emph{random port}. Notice that in this
discussion, only UDP transmission will be considered.

\subsection{Traffic filtering}
Lets suppose a team in which, for the sake of simplicity, there is
only one external (public) peer ${\cal P}^t_e$, and that a new
internal (private) peer ${\cal P}^t_i$ has sent the sequence of
[$\mathtt{hello}$]'s (see Sec~\ref{sec:peer_joining}. Lets denote
${\cal P}^t_i$'s NAT as $\mathsf{A}$. When no filtering is used at
all, $\mathsf{A}$ forwards to ${\cal P}^t_i$ any external packet that
arrives to it (obviously, if it was sent to the entry in
$\mathsf{A}$'s translation table that was created during the
transmission of the sequence of [$\mathtt{hello}$]'s), independently
on the source end-points of the packets. In the case of source (IP)
address filtering, $\mathsf{A}$ will forward the packets only if they
come from ${\cal P}^t_e$'s host.  When source end-point filtering is
used, $\mathsf{A}$ also checks the source port, i.e., that the packets
were originated at ${\cal P}^t_e$'s end-point.

\subsection{Cone VS symmetric}
Cone NATs use the same external end-point for every packet that comes
from the same internal end-point, independently on the destination of
the packets (see Fig.~\ref{fig:cone}). For the external peer ${\cal
  P}^t_e$, the situation is identical to the case in which the NATed
peer ${\cal P}^t_i$ would be running in a public host.

Symmetric NATs use different external end-points for different packets
that comes from the same internal end-point, when these packets have
different destination end-points (see Fig.~\ref{fig:symmetric}). Thus,
two different external peers will see two different public end-points
of ${\cal P}^t_e$.

\subsection{Port allocation}
In the case of port preservation, if $X$:$Y$ is the private end-point
(IP address:port) of a UDP packet, the NAT will use the public port
$Y$, if available (notice that $Y$ cound have been assigned to a
previous communcation). If $Y$ were unavailable, the NAT usually will
assign the closer free port (this is called ``sequentially port
allocation''), usually by increasing the port value, although this
behavior has not been standarized at all.

When random port allocation is implemented, the public port will be
assigned at random. Notice that, even in SN-PPA configurations, in
most of the real situations (where peers must compete with the rest of
processes that use the network for the same NAT resources,) some kind
of randomization should be always expected during a the port
assignment.

\subsection{NAT type analysis}
An incoming peer ${\cal P}^t_i$ can determine its NAT behavior using
the following steps:
\begin{enumerate}
\item Let $\{\mathsf{A}_0, \mathsf{A}_1, \cdots, \mathsf{A}_M\}$ the
  public ports used by peer ${\cal P}^t_i$, whose NAT is $\mathsf{A}$,
  to send the [$\mathtt{hello}$] UDP packets towards ${\cal S}^t$ and
  the $M$ monitor peers of the team, in this order. This data is known
  by ${\cal P}^t_i$ after receiving the acknowledgment of each
  [$\mathtt{hello}$]. Compute
  \begin{equation}
    \Delta_k = \mathsf{A}_k - \mathsf{A}_{k-1}
    \label{eq:port_distancies}
  \end{equation}
  for $k=1,2,\cdots,M$, the \emph{port distances} gathered by ${\cal
    P}^t_i$.
\item Determine a \emph{port step}
  \begin{equation}
    \Delta = \left\{\begin{array}{lr}
    0, & \text{if } \forall i, \Delta_i = 0 \\
    \mathrm{GCD}(\Delta_1, \cdots, \Delta_m), & \text{otherwise}
    \end{array}\right.
    \label{eq:port_step}
  \end{equation}
  where GCD is the Greatest Common Divisor operator.
\item If $\Delta=0$ ($\mathsf{A}$ is using the same external port for
  communicating ${\cal P}^t_i$ with the rest of peers of the team)
  then ${\cal P}^t_i$ is behind a cone NAT. Notice that public (not
  NATed) peers will be considered as being using this type of NAT,
  also.
\item If $\Delta>0$ ($\mathsf{A}$ is using a different external port
  for each external peer) then ${\cal P}^t_i$ is behind a symmetric
  NAT. In this case:
  \begin{enumerate}
  \item If
    \begin{equation}
      \Delta_1 = \Delta_2 = \cdots = \Delta_m
    \end{equation}
    then $\mathsf{A}$ is using sequentially port allocation.
  \item If
    \begin{equation}
      \Delta = \lim_{m\to\infty} \mathrm{GCD}(\Delta_1, \cdots, \Delta_m) = 1.
    \end{equation}
    then $\mathsf{A}$ is using random port allocation.
  \end{enumerate}
\end{enumerate}

\subsection{(Theoretical) NAT traversal performance of DBS}

\begin{table}
  \centering
  \begin{tabular}{l|r|r|r|r}
    Peer1/2 & CN    & CN-AF & CN-EF & SN-PPA & SN-RPA \\
    \hline
    CN      & DBS   & DBS   & DBS   & DBS    & DBS    \\
    CN-AF   & DBS   & DBS   & DBS   & NTS    & -      \\
    CN-EF   & DBS   & DBS   & DBS   & NTS    & -      \\
    SN-PPA  & DBS   & NTS   & NTS   & NTS    & -      \\
    SN-RPA  & DBS   & -     & -     & -      & -
  \end{tabular}
  \caption{NAT traversal success for different NAT typical
    combinations. CN-NF (also known by ``full cone NAT'') stands for
    Cone NAT (without packet filtering). CN-AF (also known as
    ``restricted cone NAT'') stands for Cone NAT with source Address
    Filtering. CN-EF (also known by ``port restricted cone NAT'')
    stands for Cone NAT source End-point Filtering. SN-PPA stands for
    Symmetric NAT Port Preservation Allocation, and no packet
    filtering has been considered. SN-RPA stands for Symmetric NAT
    Random Port Allocation, and no packet filtering has been used.
    \label{tab:theoretical}}
\end{table}

Table~\ref{tab:theoretical} shows the theoretical traversal success of
DBS (or an extension of it) for different NAT type combinations. Peer1
represents to a peer already joined to the team, and Peer2 to an
incoming peer. The entries labeled with ``DBS'' are those that will be
handled by DBS, out-of-the-box. An explanation of why the DBS
handshake works for such configurations is shown in
Fig.~\ref{fig:UDP-Hole-Punching-RCN}.  Notice that source end-point
filtering has been used in this example, although a similar results
should be obtained for simple source address filtering. On the other
hand, the combinations labeled with ``-'' or ``NTS'' will not work
with DBS (see Fig.\ref{fig:UDP-Hole-Punching_SN_failure}). In fact,
only the ``NTS'' entries should work, in general, with NTS, depending
on the port prediction algorithm and the number of
tries. Fig.~\ref{fig:SNTuCPP-UHP} shows an example of an NTS (NAT
traversal) success.

Summarizing, NTS can provide connectivity for those peers that are
behind port-preservation symmetric NATs with sequential port
allocation.

\subsection{A port prediction algorithm}
When both peers, Peer1 and Peer2, are behind symmetric NATs, both must
predict the port that the NAT of the interlocutor peer will use to
send the packets towards it. And obviously, this must be performed by
each already incorporated peer that is behind a symmetric NAT.

The list of predicted ports $Z$ that a a peer ${\cal P}^j_x$ performs
is determined by:
\begin{equation}
  \begin{array}{rcl}
    Z & = & \textsf{A}_0 + x + \{s\in\{0,1,\cdots,N/2-1\}\}; \\
    Z & += & \textsf{A}_0 + (x + \{s\in\{0,1,\cdots, N-1\}\}) \cdot \Delta.
  \end{array}
\end{equation}
where ``$+=$'' denotes the concatenation of lists and $N$ is the
number of guessed ports, $\textsf{A}_0$ is the first external port (the
port used to communicate with $S$) assigned to the incoming peer and
$\Delta$ is the (maximum) port step measured for the incoming peer's
NAT.

% ---------





\subsection{DBS over symmetric NATs}


Lets suppose now, that a new private peer ${\cal P}^t_j$ joins the
team through a symmetric NAT $\mathtt{B}$ (see
Fig.~\ref{fig:NATing_context}). In this case, $\mathtt{B}$ assigns
to ${\cal P}^t_j$ two different ports (and therefore, different
public end-points) $\mathsf{B}_e$ (for ${\cal P}^t_e$) and
$\mathsf{B}_i$ (for ${\cal P}^t_i$).

Notice that only one peer (${\cal P}^t_j$) is behind a symmetric NAT

if one of the peers is
behind a symmetric NAT (${\cal P}^j_i$ in our example), the DBS
handshake should also work.

Lets suppose finally, that $\mathtt{A}_j$ is also a symmetric NAT.

 However, when both peers (${\cal P}^j_l$ and
${\cal P}^j_k$, for example) run behind (different) symmetric NATs
(see Fig.\ref{fig:UDP-Hole-Punching_SN_failure}), the port-allocation
algorith dificults (and in some cases, even inhibits) the
communication between the peers. In this situation, the communication
will be success, if and only if, the $[\mathtt{hello}]$ messages are
sent by ${\cal P}^j_i$ (the NATed incoming peer) to the external
end-point that the other NAT would use to communicate with the
incomming peer, and, at the same time, the older peer creates the
corresponding translation entry in its NAT. Obviously, this will be
possible only if both peers are able to force, or at least predict,
the public ports that their NATs will use for the
$[\mathtt{hello}]$'s. Usually, peers can not configure their NAT
entries. Therefore, only the second option is available and some kind
of port prediction procedures becomes necessary.

\subsection{Port allocation}
Port prediction techniques depends on the port allocation algorithm
used by the symmetric NAT. At least 2 different strategies has been
implemented: (1) port-preservation allocation, (2) random port
allocation. In the first case, if $X:Y$ is the private end-point of a
UDP packet, the NAT will use the public port $Y$, if available ($Y$
cound have been assigned to a previous communcation). If $Y$ were
unavailable, the NAT usually will assign the closer free port, usually
in an incremental mode, although this has not been standarized. In the
second case, the public port will be assigned at random.


\subsection{Port prediction}
