% NTS

Most peers run inside of ``private'' networks, behind NAT boxes. In
this case, all peers behind the same NAT will use the same external
(also called ``public'') IP address of the NAT. Basically, there exist
two different types of NATs: (1) cone, and (2) symmetric. At the same
time, NATs can implement different strategies for filtering the
incoming public packets: (a) no filtering, (b) source IP filtering,
and (c) source end-point filtering. Finally, NATs can use several port
allocation algorithms, among which, the most frequent are: (i) port
preservation, (ii) random port allocation. Notice that in this
discussion, only UDP transmission will be considered.

\subsection{Traffic filtering}
Lets suppose a team that, for the sake of simplicity, has only one
external (public) peer ${\cal P}^t_e$, and that a new internal
(private) peer ${\cal P}^t_i$ has received the list of peers, and has
sent the sequence of [$\mathtt{hello}$]'s. Lets denote ${\cal
  P}^t_i$'s NAT as $\mathtt{A}_i$. When no filtering is used at all,
$\mathtt{A}_i$ forwards to ${\cal P}^t_i$ any external packet that
arrives to it (obviously, received through the port assigned to the
entry in $\mahttt{A}_i$'s translation table, that was created during
the transmission of the sequence of [$\mathtt{hello}$]'s), not only
coming from ${\cal P}^t_e$. In the case of source IP filtering,
$\mathtt{A}_i$ will forward the packets only if they come from ${\cal
  P}^t_e$'s host.  When source end-point filtering is used,
$\mathtt{A}_i$ also checks the source port, i.e., that the packets
were originated at ${\cal P}^t_e$'s end-point.

\subsection{Cone VS symmetric}
Cone NATs use the same external end-point for every outgoing UDP
packet that comes from the same internal end-point, independently on
the destination of the packets (see Fig.~\ref{fig:cone}). For an
external peer ${\cal P}^t_e$, the situation is identical to the case
in which the NATed peer ${\cal P}^t_i$ would be running in an external
host. In this situation, DBS should be able to establish a (UDP)
communication among cone-NATed peers, independently on the packet
filtering algorithm implemented. An example of this behavior is shown
in Fig.~\ref{fig:UDP-Hole-Punching-RCN}. Notice that source end-point
filtering has been used in this example.

Lets suppose now, that a new public peer ${\cal P}^t_j$ joins the team
through a symmetric NAT $\mathtt{A}_j$ (see
Fig.~\ref{fig:NATing_context}). In this case, $\mathtt{A}_j$ assigns
to ${\cal P}^t_j$ two different public ports (and therefore, different
public end-points) $\mathsf{A}_e$ (for ${\cal P}^t_e$) and
$\mathsf{A}_i$ (for ${\cal P}^t_i$). Only if one of the peers is
behind a symmetric NAT (${\cal P}^j_i$ in our example), the DBS
handshake should also work.

Lets suppose finally, that $\mathtt{A}_j$ is also a symmetric NAT.

 However, when both peers (${\cal P}^j_l$ and
${\cal P}^j_k$, for example) run behind (different) symmetric NATs
(see Fig.\ref{fig:UDP-Hole-Punching_SN_failure}), the port-allocation
algorith dificults (and in some cases, even inhibits) the
communication between the peers. In this situation, the communication
will be success, if and only if, the $[\mathtt{hello}]$ messages are
sent by ${\cal P}^j_i$ (the NATed incoming peer) to the external
end-point that the other NAT would use to communicate with the
incomming peer, and, at the same time, the older peer creates the
corresponding translation entry in its NAT. Obviously, this will be
possible only if both peers are able to force, or at least predict,
the public ports that their NATs will use for the
$[\mathtt{hello}]$'s. Usually, peers can not configure their NAT
entries. Therefore, only the second option is available and some kind
of port prediction procedures becomes necessary.

\subsection{Port allocation}
Port prediction techniques depends on the port allocation algorithm
used by the symmetric NAT. At least 2 different strategies has been
implemented: (1) port-preservation allocation, (2) random port
allocation. In the first case, if $X:Y$ is the private end-point of a
UDP packet, the NAT will use the public port $Y$, if available ($Y$
cound have been assigned to a previous communcation). If $Y$ were
unavailable, the NAT usually will assign the closer free port, usually
in an incremental mode, although this has not been standarized. In the
second case, the public port will be assigned at random.

% NAT type anslysis
\subsection{NAT type analysis}
A incoming peer ${\cal P}^j_i$ can determine its NAT behavior using
the following procedure:
\begin{enumerate}
\item Let $\{\mathsf{A}_0, \mathsf{A}_1, \cdots, \mathsf{A}_M\}$ the
  public ports used by ${\cal P}^j_i$'s NAT to send the
  [$\mathtt{hello}$] UDP packets towards ${\cal S}^j$ and the $M$
  monitor peers of the team, in this order. This data is received by
  ${\cal P}^j_i$ as a acknowledgement of the
  [$\mathtt{hello}$]. Compute
  \begin{equation}
    \Delta_i = \mathsf{A}_i - \mathsf{A}_{i-1}
    \label{eq:port_distancies}
  \end{equation}
  for $i=1,2,\cdots,M$, the port distances gathered by ${\cal P}^j_i$.
\item Determine a \emph{port step}
  \begin{equation}
    \Delta = \left\{\begin{array}{lr}
    0, & \text{if } \forall i, \Delta_i = 0 \\
    \mathrm{GCD}(\Delta_1, \cdots, \Delta_m), & \text{otherwise}
    \end{array}\right.
    \label{eq:port_step}
  \end{equation}
  where GCD is the Greatest Common Divisor operator.
\item If $\Delta=0$ (the NAT is using the same external port for
  communicating $P_i$ with the rest of peers of the team) then $P_i$
  is behind a cone NAT. Notice that public (not NATed) peers will be
  considered as being using this type of NAT, also.
\item If $\Delta>0$ (the NAT is using a different external port for
  each external peer) then $P_i$ is behind a symmetric NAT. In this
  case:
  \begin{enumerate}
  \item If
    \begin{equation}
      \Delta_1 = \Delta_2 = \cdots = \Delta_m
    \end{equation}
    then $P_i$'s NAT is using SPA.
  \item If
    \begin{equation}
      \Delta = \lim_{m\to\infty} \mathrm{GCD}(\Delta_1, \cdots, \Delta_m) = 1.
    \end{equation}
    then $P_i$'s NAT is using RPA.
  \end{enumerate}
\end{enumerate}

\subsection{Port prediction}
The list of predicted ports that a the peer $P_x$ performs is
determined by:
\begin{equation}
  \begin{array}{rcl}
    Z_x & = & \textsf{A}_0 + x + \{s\in\{0,1,\cdots,N/2-1\}\}; \\
    Z_x & += & \textsf{A}_0 + (x + \{s\in\{0,1,\cdots, N-1\}\}) \cdot \Delta.
  \end{array}
\end{equation}
where ``$+=$'' denotes the concatenation of lists and $N$ is the
number of guessed ports, $\textsf{A}_0$ is the first external port (the
port used to communicate with $S$) assigned to the incoming peer and
$\Delta$ is the (maximum) port step measured for the incoming peer's
NAT.
