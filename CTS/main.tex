\label{sec:CTS}
\begin{figure*}
  \fig{600}{6cm}{CTS} \caption{Tasks involved in the process of
    determing the corresponding team tracker of a given
    channel.\label{fig:CTS}}
\end{figure*}
P2PSP supposes that there is a collection of channels that are
broadcasted in parallel.\footnote{The concept of channel in P2PSP is
  similar to DVB (Digital Video Broadcasting): a stream of media that
  is transmitted in parallel with many other channels.} The channels
are generated by one or more streaming servers, and each channel has a
different URL (Universal Resource Locator), usually expressed as a
HTTP address.

When a user runs a P2PSP's media player, it connects to a channels
tracker. The tracker returns the list of channels currently
broadcasted, and the user must chose one of them. This information is
sent back to the channels tracker which responds with the end-point of
a teams tracker. Then, the player connects to the teams trackerw hich
returns the end-point of a splitter. With this information (where the
splitter is listening to the peers of its team), the player runs a
peer and the retrieving of the media stream starts.

The Fig.~\ref{fig:CTS} shows the main tasks related with CTS. As can
be seen, to find a splitter succesfully, up to four pieces of code
must be run. One function in the Player, controlled by the user, to
obtain the Splitter's end-point. One (concurrent) task in the channels
tracker that listen to the players and serve, first the list of
current channels and second, a teams tracker dedicated to such
channel. Two tasks in the teams tracker, one for serving the splitter
and another for keep updated the list of splitters with room in its
teams.
