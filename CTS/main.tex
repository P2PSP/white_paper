\label{sec:CTS}
\begin{figure*}
  \fig{800}{8cm}{CTS} \caption{Tasks involved in the process of
    determing the corresponding team tracker of a given
    channel.\label{fig:CTS}}
\end{figure*}
P2PSP supposes that there is a collection of channels that are
broadcasted in parallel.\footnote{The concept of channel in P2PSP is
  similar to DVB (Digital Video Broadcasting): a stream of media that
  is transmitted in parallel with many other channels.} The channels
are generated by one or more streaming servers, and each channel has a
different URL (Universal Resource Locator), usually expressed as a
HTTP address.

When a user runs a P2PSP's media player, it connects to a channels
tracker (see Fig.~\ref{fig:CTS}). The tracker returns the list of
channels currently broadcasted, and the user must chose one of
them. This information is sent back to the channels tracker, which
responds with the end-point of a teams tracker. Then, the player
connects to the teams tracker, that returns the end-point of a
splitter. With this information (where a splitter should be listening
to the peers of its team), the player runs a peer and the retrieving
of the media stream starts.

As can be seen in Fig.~\ref{fig:CTS}, to find a splitter succesfully,
several entities must run different tasks. The player, controlled by
the user, must obtain a splitter's end-point. In the channels tracker,
a task listen to the players and serve, first the list of current
channels and second, a teams tracker dedicated to such channel. Two
tasks in the teams tracker, one for serving the splitters and another
for keeping updated the list of splitters with room in its teams, must
be run.
