LRS extends DBS (or an extension of it) to retransmit massively-lost
chunks. LRS should be implemented if error-prone communications are
expected, specially if these channels are used by the splitter. LRS is
based on the use of monitors (see Procedure~\ref{dbs:frcS}). The idea
is: ${\cal S^j}$ will resend lost chunks to one or more the monitors
when all monitors in ${\cal T}^j$ report their loss. To increase the
probability of receiving on time the resent chunk (by normal peers),
monitors halves the number of chunks in their buffers in relation to
common peers. LRS only modifies the behavior of ${\cal S}^j$ and the
monitors. Normal peers implement DBS (or its extension).

\begin{comment}
LRS should be used when it is expected to have loss-prone
communication links or when the bit-rate of the stream exceeds the
uploading capacity of the nodes. Notice that the impact of using LSR
for normal peers is null.

P2PSP relies in UDP to transmit the chunks, and obviously, packet
losses can happen. The impact of a packet loss in the QoS provided
depends on where the packets are lost and who lost it. If a packet is
lost in its trip between the splitter and a peer, this packet will be
missed by all the team.

\subsubsection{Monitor peers}
The overlay administrator can create some special peers called, in
short,\emph{monitors}. These peers are introduced to the rest of peers
of the team as normal peers and behave like the rest of well-intended
peers, churn included. Monitor peers can be asked by the splitter to
check whether the operation of the team has and is been correct.

\subsubsection{Selfish peers rejection}
Another functionality of monitors is to tell the splitter which chunks
have not been received on time (the consequence of receiving a lost
too late is the same than not receiving it never). The splitter
memorizes which chunk has been sent to each peer in the current
round. If all monitors reports that a chunk has been lost, the
splitter can determine which peer has been in charge of broadcasting
that chunk and take note of this. If this problem happens a given
number of times, the splitter can expell the unreliable peer from the
team by stop sending to it futher chunks.

\subsubsection{Lost chunk complains} A chunk is considered as lost when
it must be played and the corresponding cell in the buffer is empty
(there is no chunk inside). More specifically, monitors send to the
splitter a [\textsf{lost chunk index} $x$] loss-report message when a
chunk with index $x==\mathsf{chunk}.\mathrm{chunk\_index}$ is lost.

\subsubsection{Retransmission of lost chunks}
Using these complaining messages, the splitter considers that the
chunk stored in the cell $x \% M$ has been lost if the number of
complains is equal to the current number of monitors in the team. In
this case, the splitter resends the lost chunk to a peer selected
randomly.

\subsubsection{Early complaining}
In order to LRS works properly, monitors must identify the lost chunks
before than the rest of peers (otherwise, the resent chunks could
arrive late to the normal peers). A simple way to achieve this is by
halving the buffer (size, measured in chunks) of the monitors. Thus,
when a monitor is reporting a lost chunk, the rest of normal peers
should be expecting to receive this chunk approximately in the middle
of their buffers, and the resent chunk should be received just on
time.
\end{comment}

\begin{comment}

  The P2PSP relies on the UDP as the transport protocol and, obviously,
packet losses can happen. The impact of a packet loss in the QoS
offered by the team depends on where the packet is lost. If the packet
is lost in the trip between the Splitter and a peer, this packet will
be missed by all the peers of the team, included the monitor
peers. However, if the packet is loss in the trip between two peers,
only the destination peer will loss the chunk. Obviously, this last
situation is less harmfull than the firt one.

Only monitor peers tell the Splitter which chunks have not been
received on time. More specifically, a monitor peer $P$ sends to the
Splitter a \texttt{[lost chunk index $x$]} loss report message when a
chunk with chunk-number $x$ has been lost. Using this information, the
Splitter can enumerate the number of times that a chunk has been
lost. In this framework, the LRS module defines that the Splitter
resend a lost chunk stored in the location $x~\text{mod}~M$ of its
buffer of chunks if the number of losses is equal to the number of
monitor peers. The selected peer to resend this block will one of the
monitor peers.

Notice also that, in this case, monitor peers should become aware of a
chunk loss some time before of that the rest of peers of the team send
it to their players, in order to have enough time to resend the lost
chunk from the monitor peer to the rest of peers of the team. A simple
technique is send \texttt{[lost chunk index $x$]} lost report messages
from the monitor peers to the Splitter when the $x$ chunk is missing
at the half of the buffer. Thus, when a monitor peer realizes that a
chunk could have been lost, the rest of peers are receiving those
chunks that are approximately in the middle of their buffers. Now, if
the buffer if large enought, the resent chunks should be received on
time. This implies also that, if the LRS module is implemented, the
buffer size should be doubled and therefore, it holds that
\begin{equation}
  B \leq 2|T|.
\end{equation}

\end{comment}
